%-----------------------------------------------------------------------------%
\chapter{\babSatu}
\label{bab:1}
%-----------------------------------------------------------------------------%
Pada bab ini, akan dijelaskan tentang latar belakang dan permasalahan yang diselesaikan pada penelitian ini.


%-----------------------------------------------------------------------------%
\section{Latar Belakang}
\label{sec:latarBelakang}
%-----------------------------------------------------------------------------%
% \todo{Tentukan latar belakang dari penelitian Anda di sini (\f{background}).}
\begin{itemize}
	\item Pentingnya algoritma dalam pengembangan perangkat lunak.
	\item Pengaruh bahasa pemrograman terhadap performa algoritma.
	\item Masalah umum: bahasa berbeda = hasil performa berbeda.
	\item Relevansi untuk dunia industri \& akademik.
\end{itemize}
\section{Rumusan Masalah}
\label{sec:rumusanMasalah}
%-----------------------------------------------------------------------------%
% \todo{Sebutkan permasalahan penelitian Anda dari latar belakang tersebut.}

\begin{itemize}
	\item Bagaimana performa beberapa algoritma klasik berbeda saat diimplementasikan pada bahasa pemrograman yang berbeda?
	\item Apa saja faktor yang mempengaruhi performa algoritma dalam konteks pengembangan perangkat lunak?
\end{itemize}

%-----------------------------------------------------------------------------%
\section{Tujuan Penelitian}
\label{sec:tujuanPenelitian}
%-----------------------------------------------------------------------------%

\begin{itemize}
	\item Menganalisis dan membandingkan performa algoritma dalam bahasa pemrograman yang berbeda
\end{itemize}

%-----------------------------------------------------------------------------%
\section{Batasan Masalah}
\label{sec:batasanMasalah}
%-----------------------------------------------------------------------------%
% Berikut ini adalah asumsi yang digunakan sebagai batasan penelitian ini:
% \begin{enumerate}
% 	\item Salah satu batasannya adalah, ini hanya \f{template}.
% \end{enumerate}

% \todo{Umumnya ada asumsi atau batasan yang digunakan untuk menjawab pertanyaan-pertanyaan penelitian diatas.}
\begin{itemize}
	\item Algoritma yang diuji: Sorting, Matrix Multiplication, BFS/DFS, Dijkstra, dll (8 total)
	\item Bahasa pemrograman: Java, Python, PHP, Rust
	\item Metrik evaluasi: waktu eksekusi, konsumsi memori (opsional: CPU usage)
	\item Eksperimen dilakukan dalam lingkungan sistem yang dikontrol
\end{itemize}


%-----------------------------------------------------------------------------%
% \section{Tujuan Penelitian}
% \label{sec:tujuan}
% %-----------------------------------------------------------------------------%
% Berikut ini adalah tujuan penelitian yang dilakukan:
% \begin{enumerate}
% 	\item Untuk memberikan \f{template} yang dapat mempermudah skripsi orang lain.
% \end{enumerate}

% \todo{Tuliskan tujuan penelitian Anda di bagian ini.}


%-----------------------------------------------------------------------------%
\section{Manfaat Penelitian}
\label{sec:manfaat}
%-----------------------------------------------------------------------------%
% Berikut ini adalah manfaat penelitian yang dilakukan:
% \begin{enumerate}
% 	\item Penelitian ini memberikan \f{template} yang harapannya dapat mempermudah skripsi orang lain.
% \end{enumerate}

% \todo{Tuliskan tujuan penelitian Anda di bagian ini.}

\begin{itemize}
	\item Memberikan referensi bagi developer/peneliti terkait pilihan bahasa dalam implementasi algoritma
	\item Menjadi acuan untuk pembelajaran pemrograman dan optimasi performa
\end{itemize}

%-----------------------------------------------------------------------------%
% \section{Posisi Penelitian}
% \label{sec:posisiPenelitian}
% %-----------------------------------------------------------------------------%
% \todo{
% 	Sebutkan posisi penelitian Anda. Ada baiknya jika Anda menggunakan gambar atau diagram.
% 	Template ini telah menyediakan contoh cara memasukkan gambar.
% }

% \begin{figure}
% 	\centering
% 	\includegraphics[width=0.4\textwidth]{assets/pics/makara.png}
% 	\caption{Penjelasan singkat terkait gambar.}
% 	\label{fig:research_position}
% \end{figure}

% \todo{
% 	Jelaskan \pic~\ref{fig:research_position} di sini.
% 	Setiap gambar yang dimasukkan ke tugas akhir \bo{WAJIB} untuk dijelaskan oleh minimal satu paragraf.
% }


%-----------------------------------------------------------------------------%
\section{Metodologi Penelitian}
\label{sec:metodologiPenelitian}
%-----------------------------------------------------------------------------%
% \todo{
% 	Subbab ini umumnya menjelaskan metode penelitian, jika metode yang digunakan cukup sederhana.
% 	Jika metode yang digunakan cukup kompleks atau ada kewajiban dari fakultas Anda untuk menjelaskan metode secara rinci, gunakan Bab 3 untuk menjelaskan Metodologi Penelitian.
% }

\begin{itemize}
	\item Implementasi algoritma di 4 bahasa
	\item Uji performa pada input beragam ukuran
	\item Bandingkan hasil berdasarkan metrik tertentu
\end{itemize}

% Berikut ini adalah langkah penelitian yang telah dilakukan:
% \begin{enumerate}
% 	\item Tinjauan literatur \\
% 	Pada tahap ini, dipelajari teori-teori yang terkait dengan penelitian ini untuk mendapatkan konsep dasar yang dibutuhkan dalam mencapai tujuan penelitian.
% 	\item Analisis implementasi dan kesimpulan \\
% 	Pada tahap ini, digunakan studi kasus untuk analisis terkait kegunaan \f{template}.
% 	Setelah melakukan analisis tersebut, ditarik kesimpulan keseluruhan dari penelitian ini.
% \end{enumerate}


%-----------------------------------------------------------------------------%
\section{Sistematika Penulisan}
\label{sec:sistematikaPenulisan}
%-----------------------------------------------------------------------------%
% Sistematika penulisan laporan adalah sebagai berikut:
% \begin{itemize}
% 	\item Bab 1 \babSatu \\
% 	    Bab ini mencakup latar belakang, cakupan penelitian, dan pendefinisian masalah.
% 	\item Bab 2 \babDua \\
% 	    Bab ini mencakup pemaparan terminologi dan teori yang terkait dengan penelitian berdasarkan hasil tinjauan pustaka yang telah digunakan, sekaligus memperlihatkan kaitan teori dengan penelitian.
% 	\item Bab 3 \babTiga \\
% 	    Apa itu Bab 3?
% 	\item Bab 4 \babEmpat \\
% 		Apa itu Bab 4?
% 	\item Bab 5 \babLima \\
% 	    Apa itu Bab 5?
% 	\item Bab 6 \kesimpulan \\
% 	    Bab ini mencakup kesimpulan akhir penelitian dan saran untuk pengembangan berikutnya.
% \end{itemize}

\begin{itemize}
	\item Bab 1: Pendahuluan
	\item Bab 2: Tinjauan Pustaka / Studi literatur
	\item Bab 3: Metodologi Penelitian
	\item Bab 4: Implementasi
	\item Bab 5: Hasil \& Evaluasi
\end{itemize}

% \todo{Anda bisa mengubah atau menambahkan penjelasan singkat mengenai isi masing-masing bab. Setiap tugas akhir pasti ada yang berbeda pada bagian ini.}
