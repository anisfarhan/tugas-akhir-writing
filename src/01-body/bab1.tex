%-----------------------------------------------------------------------------%
\chapter{\babSatu}
\label{bab:1}
%-----------------------------------------------------------------------------%
Pada bab ini, akan dijelaskan tentang latar belakang dan permasalahan yang diselesaikan pada penelitian ini.


%-----------------------------------------------------------------------------%
\section{Latar Belakang}
\label{sec:latarBelakang}
%-----------------------------------------------------------------------------%
\todo{Tentukan latar belakang dari penelitian Anda di sini (\f{background}).}

Tugas Akhir merupakan salah satu kegiatan yang menjadi "penentu nasib" atas lulusnya seorang mahasiswa dari suatu universitas.
Umumnya, mahasiswa yang mengerjakan Tugas Akhir diwajibkan untuk menuliskan sebuah laporan dengan tata bahasa ilmiah dan dengan ketentuan \f{format} tertentu.
Laporan tersebut, memiliki struktur yang kurang lebih mirip, dari manapun fakultas asal seorang mahasiswa.

Umumnya, mahasiswa Universitas Indonesia menggunakan Microsoft Word, OpenOffice, atau \f{rich-text editor} lainnya untuk menyusun laporan Tugas Akhir mereka.
Akan tetapi, terkadang \f{template} yang dibuat untuk Microsoft Word memiliki beberapa keterbatasan dan inkonsistensi dalam pengaturan tata letak.
Selain itu, terdapat beberapa hal yang sulit untuk diotomatisasi, contohnya adalah penggunaan sitasi dan pengaturan nomor halaman.
Oleh karena itu, diperlukan alternatif lain untuk menyusun Tugas Akhir yang lebih nyaman.

Beberapa mahasiswa Fakultas Ilmu Komputer Universitas Indonesia (Fasilkom UI) di tahun 2010-an mulai menggunakan \gls{latex} untuk menyusun Tugas Akhir.
Sifat \gls{latex} yang berupa \f{mark up language}, memiliki cara pemakaian yang jauh berbeda dibandingkan \f{rich-text editor} seperti Microsoft Word.
Awalnya, menyusun tata letak yang sesuai dengan persyaratan Tugas Akhir di UI cukup sulit, karena artinya harus membaca dokumentasi.
Akan tetapi, dengan kemahiran mereka dalam menyusun kode yang mudah dibaca dan terstruktur, struktur yang mereka buat dapat dijadikan sebuah \f{template} yang konsisten tata letaknya hingga bertahun-tahun berikutnya.
Dengan menggunakan sebuah \f{template}, mahasiswa cukup mengetikkan isi dari Tugas Akhir mereka tanpa harus repot mengatur ulang tata letak secara manual.
Mahasiswa bisa menggunakan \f{macro} atau perintab yang sudah disediakan untuk \f{template}.

Adanya perubahan pada peraturan Tugas Akhir yang ditentukan oleh \cite{ui:pedoman_ta}, memunculkan kebutuhan untuk perubahan \f{template} \gls{latex} ini secara signifikan.
Perubahan pada \f{template} ini dilakukan pertama kali pada semester gasal tahun ajaran 2019/2020.
Perubahan dilakukan mengingat adanya kebutuhan dari mahasiswa Lab \f{Reliable Software Engineering} (RSE), Fasilkom UI untuk menyusun Tugas Akhir menggunakan \gls{latex}.
\f{Template} ini kemudian di-\f{maintain} secara berkala, mengikuti umpan balik dan pendapat pengguna \f{template} ini, tidak hanya dari Fasilkom UI, namun juga dari berbagai fakultas lain di UI.
Dengan mulai meluasnya penggunaan \f{template} Tugas Akhir UI ini, diperlukan sebuah contoh penggunaan untuk memudahkan mahasiswa yang baru mulai melaksanakan Tugas Akhir mereka.
Dokumen dan repositori ini, diharapkan selain menjadi repositori kode \f{template}, juga menjadi repositori contoh penggunaan \f{template} dan panduan penggunaan \f{template}.



%-----------------------------------------------------------------------------%
\section{Permasalahan}
\label{sec:masalah}
%-----------------------------------------------------------------------------%
\todo{Sebutkan permasalahan penelitian Anda dari latar belakang tersebut.}

Latar belakang yang telah dijelaskan pada \sect~\ref{sec:latarBelakang}, menjelaskan bahwa diperlukan sebuah \f{template}~\gls{latex} Tugas Akhir UI yang konsisten dan bisa disesuaikan dengan peraturan terbaru dari universitas.
Selain itu, diperlukan juga sebuah \f{tutorial} (panduan penggunaan) yang memudahkan pengguna \f{template} di masa yang akan datang.

%-----------------------------------------------------------------------------%
\subsection{Definisi Permasalahan}
\label{sec:definisiMasalah}
%-----------------------------------------------------------------------------%
Berikut ini adalah rumusan permasalahan dari penelitian yang dilakukan:
\begin{enumerate}
	\item Bagaimana cara menyusun \f{template}~\gls{latex} Tugas Akhir di Universitas Indonesia?
	\item Bagaimana cara mengajarkan penggunaan \f{template} dan bahasa \gls{latex} secara umum ke mahasiswa yang akan menjalani Tugas Akhir?
\end{enumerate}
\todo{Tuliskan permasalahan yang ingin diselesaikan. Bisa juga berbentuk pertanyaan}

%-----------------------------------------------------------------------------%
\subsection{Batasan Permasalahan}
\label{sec:batasanMasalah}
%-----------------------------------------------------------------------------%
Berikut ini adalah asumsi yang digunakan sebagai batasan penelitian ini:
\begin{enumerate}
	\item Salah satu batasannya adalah, ini hanya \f{template}.
\end{enumerate}

\todo{Umumnya ada asumsi atau batasan yang digunakan untuk menjawab pertanyaan-pertanyaan penelitian diatas.}


%-----------------------------------------------------------------------------%
\section{Tujuan Penelitian}
\label{sec:tujuan}
%-----------------------------------------------------------------------------%
Berikut ini adalah tujuan penelitian yang dilakukan:
\begin{enumerate}
	\item Untuk memberikan \f{template} yang dapat mempermudah skripsi orang lain.
\end{enumerate}

\todo{Tuliskan tujuan penelitian Anda di bagian ini.}


%-----------------------------------------------------------------------------%
\section{Manfaat Penelitian}
\label{sec:manfaat}
%-----------------------------------------------------------------------------%
Berikut ini adalah manfaat penelitian yang dilakukan:
\begin{enumerate}
	\item Penelitian ini memberikan \f{template} yang harapannya dapat mempermudah skripsi orang lain.
\end{enumerate}

\todo{Tuliskan tujuan penelitian Anda di bagian ini.}


%-----------------------------------------------------------------------------%
\section{Posisi Penelitian}
\label{sec:posisiPenelitian}
%-----------------------------------------------------------------------------%
\todo{
	Sebutkan posisi penelitian Anda. Ada baiknya jika Anda menggunakan gambar atau diagram.
	Template ini telah menyediakan contoh cara memasukkan gambar.
}

\begin{figure}
	\centering
	\includegraphics[width=0.4\textwidth]{assets/pics/makara.png}
	\caption{Penjelasan singkat terkait gambar.}
	\label{fig:research_position}
\end{figure}

\todo{
	Jelaskan \pic~\ref{fig:research_position} di sini.
	Setiap gambar yang dimasukkan ke tugas akhir \bo{WAJIB} untuk dijelaskan oleh minimal satu paragraf.
}


%-----------------------------------------------------------------------------%
\section{Langkah Penelitian}
\label{sec:langkahPenelitian}
%-----------------------------------------------------------------------------%
\todo{
	Subbab ini umumnya menjelaskan metode penelitian, jika metode yang digunakan cukup sederhana.
	Jika metode yang digunakan cukup kompleks atau ada kewajiban dari fakultas Anda untuk menjelaskan metode secara rinci, gunakan Bab 3 untuk menjelaskan Metodologi Penelitian.
}
Berikut ini adalah langkah penelitian yang telah dilakukan:
\begin{enumerate}
	\item Tinjauan literatur \\
	Pada tahap ini, dipelajari teori-teori yang terkait dengan penelitian ini untuk mendapatkan konsep dasar yang dibutuhkan dalam mencapai tujuan penelitian.
	\item Analisis implementasi dan kesimpulan \\
	Pada tahap ini, digunakan studi kasus untuk analisis terkait kegunaan \f{template}.
	Setelah melakukan analisis tersebut, ditarik kesimpulan keseluruhan dari penelitian ini.
\end{enumerate}


%-----------------------------------------------------------------------------%
\section{Sistematika Penulisan}
\label{sec:sistematikaPenulisan}
%-----------------------------------------------------------------------------%
Sistematika penulisan laporan adalah sebagai berikut:
\begin{itemize}
	\item Bab 1 \babSatu \\
	    Bab ini mencakup latar belakang, cakupan penelitian, dan pendefinisian masalah.
	\item Bab 2 \babDua \\
	    Bab ini mencakup pemaparan terminologi dan teori yang terkait dengan penelitian berdasarkan hasil tinjauan pustaka yang telah digunakan, sekaligus memperlihatkan kaitan teori dengan penelitian.
	\item Bab 3 \babTiga \\
	    Apa itu Bab 3?
	\item Bab 4 \babEmpat \\
		Apa itu Bab 4?
	\item Bab 5 \babLima \\
	    Apa itu Bab 5?
	\item Bab 6 \kesimpulan \\
	    Bab ini mencakup kesimpulan akhir penelitian dan saran untuk pengembangan berikutnya.
\end{itemize}

\todo{Anda bisa mengubah atau menambahkan penjelasan singkat mengenai isi masing-masing bab. Setiap tugas akhir pasti ada yang berbeda pada bagian ini.}
