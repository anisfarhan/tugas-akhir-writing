%-----------------------------------------------------------------------------%
\chapter{\babDua}
\label{bab:2}
%-----------------------------------------------------------------------------%
\todo{
	Bab ini, biasanya namanya adalah "Studi Literatur" atau "Tinjauan Pustaka".
	Akan tetapi, beberapa fakultas atau dosen pembimbing meminta Bab 2 untuk dinamakan lain, seperti "Kerangka Berpikir".
}

% Untuk memulai penelitian, dibutuhkan kerangka berpikir yang sesuai untuk permasalahan yang ingin dipecahkan.
% Untuk membentuk kerangka berpikir yang sesuai, perlu dikaitkan dengan hasil studi literatur yang telah dilakukan.
% Oleh karena itu, pada bab ini, akan dijelaskan hasil studi literatur yang telah dilakukan yang telah dikaitan dengan kerangka kerja untuk penelitian ini.

\section{Algoritma yang digunakan}
\label{sec:algoritma}
\begin{itemize}
	\item Sorting (Bubble Sort): logika, kompleksitas
	\item Matrix Multiplication: metode naïve O(n³)
	\item Graph Traversal (DFS & BFS): konsep & aplikasi
	\item Dijkstra: shortest path
	\item Knapsack/Coin Change: dynamic programming
	\item String Matching: naive/KMP
	\item Path-finding: konsep umum
	\item Huffman Coding: algoritma kompresi berbasis frekuensi
\end{itemize}

\section{Bahasa Pemrograman yang digunakan}
\label{sec:bahasaPemrograman}
\begin{itemize}
	\item Java: bahasa compiled, OOP kuat
	\item Python: interpreted, sintaks sederhana, banyak dipakai di AI
	\item PHP: scripting web, masih digunakan luas
	\item Rust: sistem modern, fokus ke performa dan safety
\end{itemize}

\section{Studi Terdahulu (Literature Review)}
\label{sec:literatureReview}
\begin{itemize}
	\item Penelitian serupa tentang perbandingan performa algoritma
	\item Studi benchmark bahasa pemrograman
	\item Relevansi penelitian dengan topik skripsi kamu
\end{itemize}

\section{Ringkasan Literatur}
\label{sec:ringkasanLiteratur}
\begin{itemize}
	\item Tabel ringkasan hasil penelitian terdahulu dan posisi penelitian kamu
\end{itemize}

%-----------------------------------------------------------------------------%
% \section{Apa itu \latex?}
% \label{sec:latex}
% %-----------------------------------------------------------------------------%

% %-----------------------------------------------------------------------------%
% \subsection{\latex~Secara Singkat}
% \label{sec:latexBrief}
% %-----------------------------------------------------------------------------%
% Berdasarkan \cite{latex:intro}: \\
% \begin{tabular}{| p{14cm} |}
% 	\hline
% 	\gls{latex} is a family of programs designed to produce publication-quality typeset documents.
% 	It is particularly strong when working with mathematical symbols. \\
% 	The history of \gls{latex} begins with a program called TEX.
% 	In 1978, a computer scientist by the name of Donald Knuth grew frustrated with the mistakes that his publishers made in typesetting his work.
% 	He decided to create a typesetting program that everyone could easily use to typeset documents, particularly those that include formulae, and made it freely available.
% 	The result is TEX. \\
% 	Knuth's product is an immensely powerful program, but one that does focus very much on small details.
% 	A mathematician and computer scientist by the name of Leslie Lamport wrote a variant of TEX called \gls{latex} that focuses on document structure rather than such details. \\
% 	\hline
% \end{tabular}

% \vspace*{0.2cm}

% Dokumen \gls{latex}~sangat mudah, seperti halnya membuat dokumen teks biasa.
% Ada beberapa perintah yang diawali dengan tanda '\bslash'.
% Seperti perintah \code{\bslash\bslash}~yang digunakan untuk memberi baris baru.
% Perintah tersebut juga sama dengan perintah \code{\bslash{}newline}.
% Pada bagian ini akan sedikit dijelaskan cara manipulasi teks dan perintah-perintah \gls{latex}~yang mungkin akan sering digunakan.
% Jika ingin belajar hal-hal dasar mengenai \gls{latex}, silakan kunjungi:

% \begin{itemize}
% 	\item \url{http://frodo.elon.edu/tutorial/tutorial/}, atau
% 	\item \url{http://www.maths.tcd.ie/~dwilkins/LaTeXPrimer/}
% \end{itemize}

% %-----------------------------------------------------------------------------%
% \subsection{\latex~Kompiler dan IDE}
% \label{sec:latexCompiler}
% %-----------------------------------------------------------------------------%
% Untuk menggunakan \gls{latex}~(pada konteks hanya sebagai pengguna), tidak perlu banyak tahu mengenai hal-hal didalamnya.
% Dengan menggunakan \f{Integrated Development Environment} (IDE), penggunaan \gls{latex}~akan serupa dengan pembuatan dokumen secara visual, layaknya OpenOffice Writer atau Microsoft Word.
% Orang-orang yang menggunakan \gls{latex}~relatif lebih teliti dan terstruktur mengenai cara penulisan yang dia gunakan, karena \gls{latex}~memaksa untuk seperti itu.

% Untuk mencoba \gls{latex}, diperlukan kompiler dan IDE.
% Bagi pengguna Microsoft Windows dan Mac OS, instalasi kompiler \gls{latex}~dapat menggunakan MikTeX (\url{https://miktex.org/download}).
% Bagi pengguna Linux, instalasi kompiler \gls{latex}~dapat menggunakan Texlive ( \url{http://www.tug.org/texlive/}).
% Distro-distro \f{mainstream} di Linux seperti Ubuntu biasanya telah menyediakan \f{package} \code{texlive} melalui \f{package manager}.
% Apabila ingin melakukan instalasi Texlive melalui \f{package manager}, lakukan instalasi package \code{texlive-full} atau setidaknya \code{texlive-science} agar prasyarat \f{template} ini tersedia secara lengkap.

% Beberapa text editor atau IDE yang dapat digunakan adalah sebagai berikut:
% \begin{itemize}
% 	\item \underline{\href{https://www.texstudio.org/}{TeXstudio}} (direkomendasikan),
% 	\item TeXWorks (biasanya bawaan dari \underline{\href{https://miktex.org/download}{MikTeX}}),
% 	\item \underline{\href{http://www.xm1math.net/texmaker/}{Texmaker}}, atau
% 	\item Microsoft Visual Studio Code, dengan \f{plugin} \underline{\href{https://marketplace.visualstudio.com/items?itemName=James-Yu.latex-workshop}{\gls{latex} Workshop}}.
% 	Untuk menggunakan \f{plugin} tersebut, diperlukan instalasi MikTeX dan Perl.
% 	Alternatif lain untuk persyaratan tersebut adalah menggunakan \f{plugin} Remote - WSL jika memiliki distro Windows Subsystem for Linux (WSL) 2 yang sudah terpasang \code{texlive}.
% \end{itemize}


% %-----------------------------------------------------------------------------%
% \section{\f{Formatting} Teks Dasar}
% \label{sec:latexBasicFormatting}
% %-----------------------------------------------------------------------------%
% Hal pertama yang mungkin ditanyakan adalah bagaimana membuat huruf tercetak tebal, miring, atau memiliki garis bawah.
% Pada Texmaker, Anda bisa melakukan hal ini seperti halnya saat mengubah dokumen dengan OO Writer.
% Namun jika tetap masih tertarik dengan cara lain, ini dia:

% \begin{itemize}
% 	\item \bo{Bold} \\
% 	Gunakan perintah \code{\bslash{}textbf$\lbrace\rbrace$} atau
% 	\code{\bslash{}bo$\lbrace\rbrace$}.\\
% 	Contoh: \textbf{Contoh hasil tulisan} atau \bo{Contoh hasil tulisan}.
% 	\item \f{Italic} \\
% 	Gunakan perintah \code{\bslash{}textit$\lbrace\rbrace$} atau
% 	\code{\bslash{}f$\lbrace\rbrace$}.\\
% 	Contoh: \textit{Contoh hasil tulisan} atau \f{Contoh hasil tulisan}.
% 	\item \underline{Underline} \\
% 	Gunakan perintah \code{\bslash{}underline$\lbrace\rbrace$}.\\
% 	Contoh: \underline{Contoh hasil tulisan}.
% 	\item $\overline{Overline}$ \\
% 	Gunakan perintah \code{\$\bslash{}overline\$}.\\
% 	Contoh: $\overline{Contoh~hasil~tulisan}$.
% 	\item $^{superscript}$ \\
% 	Gunakan perintah \code{\$\bslash{}$\lbrace\rbrace$\$}.\\
% 	Contoh: $^{Contoh~hasil~tulisan}$.
% 	\item $_{subscript}$ \\
% 	Gunakan perintah \code{\$\bslash{}\_$\lbrace\rbrace$\$}.\\
% 	Contoh: $_{Contoh~hasil~tulisan}$.
% \end{itemize}

% % v2.2.1: tutorial dipindah dari Bab 3 ke Bab 2
% Ada beberapa hal lain yang bisa digunakan.
% \begin{itemize}
% 	\item Kombinasi \bo{Bold} dan \f{Italic}: \\
% 	Gunakan perintah \code{\bslash{}bi$\lbrace\rbrace$}. \\
% 	Contoh: \bi{Contoh hasil tulisan}.
% 	\item Menebalkan teks formula matematis: \\
% 	Gunakan perintah \code{\bslash{}m$\lbrace\rbrace$}. \\
% 	Contoh: \m{\alpha~\beta}
% 	\item Menebalkan teks formula matematis, sekaligus meletakkannya di tengah: \\
% 	Gunakan perintah \code{\bslash{}mc$\lbrace\rbrace$}. \\
% 	Contoh: \mc{\alpha~\beta}
% 	\item Menggunakan \f{monospaced font} untuk kode:
% 	Gunakan perintah \code{\bslash{}texttt$\lbrace\rbrace$} atau \code{\bslash{}code$\lbrace\rbrace$}. \\
% 	Contoh: \texttt{Contoh hasil tulisan} atau \code{Contoh hasil tulisan}.
% \end{itemize}

% Perintah \code{\bslash{}f}, \code{\bslash{}bo}, \code{\bslash{}bi}, \code{\bslash{}m}, \code{\bslash{}mc},
% dan \code{\bslash{}code} hanya dapat digunakan jika package \code{\_internals/uithesis} digunakan.

% %-----------------------------------------------------------------------------%
% \section{Memasukan Gambar}
% \label{sec:latexImage}
% %-----------------------------------------------------------------------------%
% Setiap gambar dapat diberikan caption dan diberikan label. Label dapat digunakan untuk menunjuk gambar tertentu.
% Jika posisi gambar berubah, maka nomor gambar juga akan diubah secara otomatis.
% Begitu juga dengan seluruh referensi yang menunjuk pada gambar tersebut.
% Contoh sederhana adalah \pic~\ref{fig:testGambar}, yang bisa dibuat dengan menggunakan \lst~\ref{code:latexImage}.
% Harap diingat pada aturan Tugas Akhir UI, caption harus selalu \b{diletakkan di bawah gambar}.

% \lstinputlisting[language={[latex]tex}, caption=Contoh penggunaan gambar, label=code:latexImage]{assets/codes/tutorial/2-basicFigure.tex}

% Berikut adalah penjelasan dari \lst~\ref{code:latexImage}:
% \begin{itemize}
% 	\item Baris ke-2: \code{\bslash{}centering} digunakan untuk membuat gambar berada di tengah.
% 	\item Baris ke-3 dan 4: \code{\bslash{}includegraphics} digunakan untuk memasukkan gambar. \\
% 	\code{width=0.25\bslash{}textwidth} digunakan untuk mengatur lebar gambar sebesar 25\% dari lebar teks (dari ujung marjin kiri ke ujung marjin kanan).
% 	\item Baris ke-5: \code{\bslash{}caption} digunakan untuk memberikan \f{caption} pada gambar.
% 	\f{Caption} tersebut diletakkan setelah \code{includegraphics} agar \f{caption} berada di bawah gambar.
% 	\item Baris ke-6: \code{\bslash{}label} digunakan untuk memberikan label pada gambar.
% 	Label ini bisa digunakan di suatu paragraf untuk merujuk pada gambar tersebut,
% 	dengan cara menuliskan \code{\bslash{}ref\{label\}} pada paragraf.
% \end{itemize}

% \begin{figure}
	\centering
	\includegraphics[width=0.25\textwidth]
	{assets/pics/makara_kuning.png}
	\caption{Makara Universitas Indonesia}
	\label{fig:testGambar}
\end{figure}


% Anda juga bisa memasukkan sitasi atau URL sumber gambar, jika gambar tersebut bukan Anda sendiri yang membuatnya.
% Contoh sederhana adalah \pic~\ref{fig:testGambarBersumber}, yang bisa dibuat dengan menggunakan \lst~\ref{code:latexImageSource}.

% \lstinputlisting[language={[latex]tex}, caption=Contoh penggunaan gambar bersumber, label=code:latexImageSource]{assets/codes/tutorial/2-sourcedFigure.tex}

% Pada baris ke-5, \code{\bslash{}captionsource} digunakan untuk memberikan caption dan sumber gambar.
% Dalam kasus ini, sumber gambar merupakan sebuah URL, sehingga ditandai dengan perintah \code{\bslash{}url\{\}}.
% Jika sumber gambar merupakan sebuah buku, jurnal, atau dokumen, maka bisa dilakukan sitasi menggunakan perintah \code{\bslash{}cite\{\}}.
% Contoh: \code{\bslash{}captionsource\{Sesuatu\}\{\bslash{}cite\{latex:intro\}\}}.

% \begin{figure}
	\centering
	\includegraphics[width=0.50\textwidth]
	{assets/pics/creative_commons.png}
	\captionsource{\license.}{{\url{https://creativecommons.org/licenses/by-nc-sa/1.0/}}}
	\label{fig:testGambarBersumber}
\end{figure}



% %-----------------------------------------------------------------------------%
% \section{Membuat Tabel}
% \label{sec:latexTable}
% %-----------------------------------------------------------------------------%
% Tabel pada \gls{latex} dapat dibuat secara visual dengan bantuan \f{website} seperti \url{https://www.tablesgenerator.com/}.
% Dengan menggunakan \textit{website} ini, maka pembuatan tabel akan menjadi lebih mudah.
% \textit{User interface} dari \f{website} dapat dilihat pada Gambar \ref{fig:tablesgenerator}.

% \begin{figure}
% 	\centering
% 	\includegraphics[width=0.5\textwidth]{assets/pics/tablesgenerator-dot-com.png}
% 	\caption{\textit{User interface} dari \textit{website} https://www.tablesgenerator.com/}
% 	\label{fig:tablesgenerator}
% \end{figure}

% Di sisi lain, tabel juga dapat diberi label dan caption seperti pada gambar.
% Caption pada tabel terletak pada bagian atas tabel.
% Contoh kode yang menyusun suatu tabel sederhana dapat dilihat pada \lst~\ref{code:latexTable}.

% \lstinputlisting[language={[latex]tex}, caption=Contoh penggunaan tabel, label=code:latexTable]{assets/codes/tutorial/2-basicTable.tex}

% Berikut adalah penjelasan dari \lst~\ref{code:latexTable}:
% \begin{itemize}
% 	\item Baris ke-2: \code{\bslash{}centering} digunakan untuk membuat tabel berada di tengah.
% 	\item Baris ke-3: \code{\bslash{}caption} digunakan untuk memberikan \f{caption} pada tabel.
% 	\f{Caption} tersebut diletakkan setelah \code{begin\{tabular\}} agar \f{caption} berada di atas tabel.
% 	\item Pada baris ke-3, terdapat argumen \code{| l | c r |} yang artinya adalah sebagai berikut:
% 	\begin{itemize}
% 		\item \code{|} digunakan untuk membuat garis vertikal pada tabel.
% 		\item \code{l} digunakan untuk membuat suatu kolom menjadi rata kiri.
% 		\item \code{c} digunakan untuk membuat suatu kolom menjadi rata tengah.
% 		\item \code{r} digunakan untuk membuat suatu kolom menjadi rata kanan.
% 	\end{itemize}
% 	\item Baris ke-4: \code{\bslash{}label} digunakan untuk memberikan label pada tabel.
% 	Label ini bisa digunakan di suatu paragraf untuk merujuk pada tabel tersebut,
% 	dengan cara menuliskan \code{\bslash{}ref\{label\}} pada paragraf.
% 	\item Baris ke-5: \code{\bslash{}begin\{tabular\}} digunakan untuk memulai pembuatan tabel.
% 	\item \code{\bslash{}hline} digunakan untuk membuat garis horizontal pada tabel.
% 	\item \code{\&} digunakan untuk memisahkan antar kolom.
% 	\item \code{\bslash{}\bslash{}} digunakan untuk memisahkan antar baris.
% 	\item Baris ke-17 dan 18: Kode untuk mengakhiri pembuatan tabel.
% \end{itemize}

% Hasil dari \lst~\ref{code:latexTable} akan menjadi \tab~\ref{tab:basic}.

% \begin{table}
	\centering
	\caption{Contoh Tabel}
	\label{tab:basic}
	\begin{tabular}{| l | c r |} %
		\hline % garis lurus horizontal
		& kol 1 & kol 2 \\ % baris 1
		\hline %
		baris 1 & 1 & 2 \\ % baris 2
		baris 2 & 3 & 4 \\ % baris 3
		baris 3 & 5 & 6 \\ % baris 4
		baris 4 & 7 & 8 \\ % baris 5
		baris 5 & 9 & 10 \\ % baris 6
		\hline
		\bo{jumlah} & \bo{25} & \bo{30} \\ % baris 7
		\hline
	\end{tabular}
\end{table}



% %-----------------------------------------------------------------------------%
% \subsection{Tabel Panjang (Lintas Baris)}
% \label{sec:latexLongTable}
% %-----------------------------------------------------------------------------%
% Adapun untuk membuat tabel panjang yang bisa melebihi dari satu halaman,
% gunakan perintah \code{\bslash{}begin\{longtable\}} sebagai pengganti \code{\bslash{}begin\{table\}}.
% Di dalam \code{longtable} tidak perlu lagi ada \code{\bslash{}begin\{tabular\}}.
% Kemudian, tambahkan tanda \code{\bslash{}\bslash{}} setelah baris \code{\bslash{}label\{....\}},
% agar tidak menimbulkan error saat menampilkan \f{caption} di bagian atas tabel.
% Kemudian, untuk membatasi header yang ingin diulang pada halaman-halaman berikutnya,
% gunakan perintah \code{\bslash{}endhead}.
% Contoh kode pembuatan tabel pankang dapat dilihat pada \lst~\ref{code:latexLongTable}.

% \lstinputlisting[language={[latex]tex}, caption=Contoh penggunaan tabel panjang, label=code:latexLongTable]{assets/codes/tutorial/2-longTable.tex}

% Terdapat lima bagian pada sebuah tabel panjang:
% \begin{enumerate}
% 	\item Awalan (\f{header}) di halaman pertama, umumnya disebut sebgai \code{firsthead}.\\
% 	Pada \lst~\ref{code:latexLongTable}, bagian ini didefinisikan di baris 2 sampai 7.
% 	Bagian ini diakhiri definisinya dengan perintah \code{\bslash{}endfirsthead}.
% 	Pada bagian ini, \f{caption} yang digunakan merupakan \f{caption} asli.
% 	\code{caption} dan \code{label} hanya perlu didefinisikan di bagian awal tabel, di halaman pertama saja.
% 	Anda juga tetap bisa menggunakan \code{captionsource} apabila dibutuhkan.
% 	\item Awalan (\f{header}) di halaman berikutnya, umumnya disebut sebagai \code{head}.\\
% 	Pada \lst~\ref{code:latexLongTable}, bagian ini didefinisikan di baris 8 sampai 12.
% 	Bagian ini diakhiri definisinya dengan perintah \code{\bslash{}endhead}.
% 	Terkait penggunaan \f{caption} sambungan:
% 	\begin{itemize}
% 		\item Jika Anda menggunakan \code{\bslash{}caption\{\}}, maka untuk menuliskan caption sambungan, gunakan perintah \code{\bslash{}caption[]\{\}}.
% 		Pada kasus \lst~\ref{code:latexLongTable}, \f{caption} sambungan didefinisikan di baris 8 menggunakan perintah \code{\bslash{}caption[]\{\}}.
% 		\item Jika Anda menggunakan \code{\bslash{}captionsource\{\}\{\}}, maka untuk menuliskan caption sambungan, gunakan perintah \code{\bslash{}captionsourcecont\{\}\{\}}.
% 	\end{itemize}
% 	\item Akhiran (\f{footer}) yang muncul di halaman pertama hingga sebelum terakhir, umumnya disebut sebagai \code{foot}.\\
% 	Bagian ini diakhiri definisinya dengan perintah \code{\bslash{}endfoot}.
% 	Pada \lst~\ref{code:latexLongTable}, bagian ini didefinisikan di baris 13 sampai 14.
% 	Dalam kasus ini, akhiran tabel hanya berupa garis horizontal.
% 	\item Akhiran (\f{footer}) di halaman terakhir, umumnya disebut sebagai \code{lastfoot}.\\
% 	Bagian ini diakhiri definisinya dengan perintah \code{\bslash{}endlastfoot}.
% 	Pada \lst~\ref{code:latexLongTable}, bagian ini didefinisikan di baris 15 sampai 16.
% 	Dalam kasus ini, akhiran tabel hanya berupa garis horizontal.
% 	\item Isi dari tabel.
% 	Pada \lst~\ref{code:latexLongTable}, bagian ini didefinisikan di baris 17 sampai 32.
% 	Isi dari tabel akan diletakkan di antara awalan tabel (\f{header}) dan akhiran tabel (\f{footer}).
% \end{enumerate}

% Hasil dari \lst~\ref{code:latexLongTable} akan menjadi \tab~\ref{tab:long}.

% \begin{longtable}{| l | c r |}
    \caption{Contoh Tabel Panjang}
    \label{tab:long} \\
    \hline
    & kol 1 & kol 2 \\
    \hline
    \endfirsthead % batas akhir header yang akan muncul di halaman pertama
    \caption[]{Contoh Tabel Panjang (sambungan)} \\
    \hline
    & kol 1 & kol 2 \\
    \hline
    \endhead % batas akhir header yang akan muncul di halaman berikutnya
    \hline
    \endfoot % batas akhir footer yang akan muncul di halaman berikutnya
    \hline
    \endlastfoot % batas akhir footer yang akan muncul di halaman terakhir
    baris 1  & 1 & 2 \\
    baris 2  & 3 & 4 \\
    baris 3  & 5 & 6 \\
    baris 4  & 7 & 8 \\
    baris 5  & 9 & 10 \\
    baris 6  & 11 & 12 \\
    baris 7  & 13 & 14 \\
    baris 8  & 15 & 16 \\
    baris 9  & 17 & 18 \\
    baris 10 & 19 & 20 \\
    baris 11 & 21 & 22 \\
    baris 12 & 23 & 24 \\
    baris 13 & 25 & 26 \\
    baris 14 & 27 & 28 \\
    baris 15 & 29 & 30 \\
    baris 16 & 31 & 32 \\
\end{longtable}


% %-----------------------------------------------------------------------------%
% \subsection{Menggabungkan (\f{Merge}) Baris atau Kolom}
% \label{sec:latexMergeCellsTable}
% %-----------------------------------------------------------------------------%
% Ada jenis tabel lain yang dapat dibuat dengan \gls{latex}~berikut beberapa diantaranya.
% Contoh-contoh ini bersumber dari \url{https://en.wikibooks.org/wiki/LaTeX/Tables}.

% \bo{Contoh 1: Menggabungkan Kolom} \\
% Contoh penggunaan tabel dengan sel yang melebar ke lebih dari satu kolom dapat dilihat pada \lst~\ref{code:mergeColumnTable}.
% Pada contoh ini, sel pada baris 1, kolom 3 dan 4 digabungkan menjadi satu dengan menggunakan perintah \code{\bslash{}multicolumn\{3\}\{c\}\{Week 1\}}.

% \lstinputlisting[language={[latex]tex}, caption=Contoh penggunaan tabel dengan sel yang melebar ke lebih dari satu kolom, label=code:mergeColumnTable]{assets/codes/tutorial/2-mergeColumnTable.tex}

% Hasil dari \lst~\ref{code:mergeColumnTable} akan menjadi \tab~\ref{tab:rowSpanning}.

% \begin{table}
	\centering
	\captionsource{Tabel dengan sel yang melebar ke lebih dari satu kolom}{\url{https://en.wikibooks.org/wiki/LaTeX/Tables}, dengan modifikasi.}
	\label{tab:rowSpanning}
	\begin{tabular}{|l|l|*{6}{c|}}
		% Baris 1
		\hline % buat garis horizontal dari kolom pertama ke kolom terakhir
		No & Name & \multicolumn{3}{|c|}{Week 1} & \multicolumn{3}{|c|}{Week 2} \\
		% Baris 2
		\cline{3-8} % buat garis horizontal dari kolom 3 sampai 8
		& & A & B & C & A & B & C\\
		% Baris 3
		\hline
		1 & Lala & 1 & 2 & 3 & 4 & 5 & 6\\
		% Baris 4
		2 & Lili & 1 & 2 & 3 & 4 & 5 & 6\\
		% Baris 5
		3 & Lulu & 1 & 2 & 3 & 4 & 5 & 6\\
		\hline
	\end{tabular}
\end{table}


% \bo{Contoh 2: Menggabungkan Baris} \\
% Contoh penggunaan tabel dengan sel yang melebar ke lebih dari satu baris dapat dilihat pada \lst~\ref{code:mergeRowTable}.
% Pada contoh ini, sel pada kolom 1, baris 2 dan 3 digabungkan menjadi satu dengan menggunakan perintah \code{\bslash{}multirow\{2\}\{*\}\{Kedua\}}.

% \lstinputlisting[language={[latex]tex}, caption=Contoh penggunaan tabel dengan sel yang melebar ke lebih dari satu baris, label=code:mergeRowTable]{assets/codes/tutorial/2-mergeRowTable.tex}

% Hasil dari \lst~\ref{code:mergeRowTable} akan menjadi \tab~\ref{tab:columnSpanning}.

% \begin{table}
	\centering
	\captionsource{Tabel dengan sel yang melebar ke lebih dari satu baris}{\url{https://en.wikibooks.org/wiki/LaTeX/Tables}, dengan modifikasi.}
	\label{tab:columnSpanning}
	\begin{tabular}{|l|c|l|} % border vertikal ditandai dengan |
        % Baris 1
		\hline % buat garis horizontal dari kolom pertama ke kolom terakhir
		Percobaan & Iterasi & Waktu \\
        % Baris 2
		\hline
		Pertama & 1 & 0.1 sec \\ \hline
        % Baris 3 (kolom 1 melebar 2 baris)
		\multirow{2}{*}{Kedua} & 1 & 0.1 sec \\
        % Baris 4 (kolom 1 dikosongkan karena sudah ditimpa)
		& 3 & 0.15 sec \\
        % Baris 5 (kolom 1 melebar 3 baris)
		\hline
		\multirow{3}{*}{Ketiga} & 1 & 0.09 sec \\
        % Baris 6 (kolom 1 dikosongkan karena sudah ditimpa)
		& 2 & 0.16 sec \\
        % Baris 7 (kolom 1 dikosongkan karena sudah ditimpa)
		& 3 & 0.21 sec \\
		\hline
	\end{tabular}
\end{table}


% \bo{Contoh 3: Menggabungkan Baris dan Kolom} \\
% Contoh penggunaan tabel dengan sel yang melebar ke lebih dari satu kolom dan baris dapat dilihat pada \lst~\ref{code:mergeBothTable}.

% \lstinputlisting[language={[latex]tex}, caption=Contoh penggunaan tabel dengan sel yang melebar ke lebih dari satu kolom dan baris, label=code:mergeBothTable]{assets/codes/tutorial/2-mergeBothTable.tex}

% Hasil dari \lst~\ref{code:mergeBothTable} akan menjadi \tab~\ref{tab:mixSpanning}.

% \begin{table}
	\centering
	\captionsource{Tabel dengan sel yang melebar ke lebih dari satu kolom dan baris}{\url{https://en.wikibooks.org/wiki/LaTeX/Tables}, dengan modifikasi.}
	\label{tab:mixSpanning}
	\begin{tabular}{|cc|c|c|c|c|} % border vertikal ditandai dengan |, semua kolom rata tengah.
        % 2-row 2-col span (rata tengah), dan 4-col span (rata tengah)
		\hline % buat garis horizontal dari kolom pertama ke kolom terakhir
		\multicolumn{2}{|c|}{\multirow{2}{*}{Element}} & \multicolumn{4}{c|}{Title} \\
        % kelanjutan border untuk 2-row 2-col span, dan 4 kolom biasa
        \cline{3-6} % buat garis horizontal dari kolom 3 sampai 6
		\multicolumn{2}{|c|}{} & A & B & C & D \\
        % 2-row span (rata kiri), dan 5 kolom biasa
        \hline
		\multicolumn{1}{|l|}{\multirow{2}{*}{Type}} & X & 1 & 2 & 3 & 4 \\
        % kelanjutan border untuk 2-row span, dan 5 kolom biasa
        \cline{2-6}
		\multicolumn{1}{|l|}{} & Y & 0.5 & 1.0 & 1.5 & 2.0 \\
        % kelanjutan bordering untuk 2-row span, dan 5 kolom biasa
        % 2-row span (rata kiri), dan 5 kolom biasa
        \hline
		\multicolumn{1}{|l|}{\multirow{2}{*}{Resource}} & I & 10 & 20 & 30 & 40\\
        % kelanjutan border untuk 2-row span, dan 5 kolom biasa
        \cline{2-6}
		\multicolumn{1}{|l|}{} & J & 5 & 10 & 15 & 20 \\
        \hline
	\end{tabular}
\end{table}




% %-----------------------------------------------------------------------------%
% \section{Membuat Persamaan Matematis}
% \label{sec:mathEqu}
% %-----------------------------------------------------------------------------%
% % v2.2.1: tutorial dipindah dari Bab 3 ke Bab 2
% Di \gls{latex}, kita dapat membuat persamaan matematis baik yang terdiri dari satu persamaan maupun lebih dari satu persamaan.
% Anda bisa mencoba mengikuti dan memahami contoh kode yang ada di \f{template} ini untuk kebutuhan tugas akhir Anda.
% Menggunakan \gls{latex}~juga perlu latihan dan lihai memahami dokumentasi.

% %-----------------------------------------------------------------------------%
% \subsection{Satu Persamaan}
% \label{sec:oneEqu}
% %-----------------------------------------------------------------------------%
% \noindent \begin{align}\label{equ:garis}
	\cfrac{y - y_{1}}{y_{2} - y_{1}} =
	\cfrac{x - x_{1}}{x_{2} - x_{1}}
\end{align}

\addequtotoc{equ:garis}{Persamaan garis}


% \equ~\ref{equ:garis} di atas adalah persamaan garis.
% Persamaan tersebut dapat disusun dengan menggunakan perintah \code{\bslash{}align}, seperti yang ditunjukkan pada \lst~\ref{code:singleEquationGaris}.
% Penggunaan perintah \code{\bslash{}noindent} sebelum memanggil \f{environment} \code{align} diperlukan agar persamaan yang dicetak tidak tergeser mengikuti indentasi yang umumnya muncul di awal paragraf.
% Terdapat beberapa perintah yang digunakan untuk menyusun \equ~\ref{equ:garis}, yaitu:
% \begin{itemize}
% 	\item Perintah \code{\bslash{}cfrac\{\}\{\}} untuk menulis pecahan dalam bentuk vertikal.
% 	Argumen pertama adalah pembilang (di atas), sedangkan argumen kedua adalah penyebut (di bawah).
% 	\item Perintah \code{\bslash{}\_\{\}} untuk menulis \f{subscript}.
% \end{itemize}

% Di luar \f{environment} \code{align}, kita juga bisa menambahkan persamaan tersebut ke daftar persamaan dengan menggunakan perintah \code{\bslash{}addequtotoc\{label\}\{caption\}}.
% Perintah \code{\bslash{}label} pada persamaan matematis hanya akan memberikan label,
% namun tidak menambahkan ke daftar persamaan.
% Perintah \code{\bslash{}addequtotoc} harus ditambahkan di luar \f{environment} \code{align},
% karena perintah tersebut harus dijalankan di mode \f{typesetting} teks biasa agar tidak terjadi \f{error}.

% \lstinputlisting[language={[latex]tex}, caption=Kode pembuatan \equ~\ref{equ:garis}, label=code:singleEquationGaris]{assets/codes/tutorial/2-singleEquationGaris.tex}

% Persamaan bola berikut, yang ditunjukkan oleh \ref{equ:bola}, merupakan contoh lain menyusun sebuah persamaan di \gls{latex}.

% \noindent \begin{align}\label{equ:bola}
	\underbrace{|\overline{ab}|}_{\text{pada bola $|\overline{ab}| = r$}}
	= \sqrt[2]{(x_{b} - x_{a})^{2} + (y_{b} - y_{a})^{2} +
		\vert\vert(z_{b} - z_{a})^{2}}
\end{align}

\addequtotoc{equ:bola}{Persamaan bola}


% Terdapat beberapa perintah yang digunakan untuk menyusun \equ~\ref{equ:bola}, yaitu:
% \begin{itemize}
% 	\item Perintah \code{\bslash{}sqrt\{\}} untuk menulis akar.
% 	\item Perintah \code{\bslash{}underbrace\{\}} untuk menulis \f{brace} di bawah suatu ekspresi.
% 		Biasanya digunakan untuk memberikan anotasi terhadap suatu ekspresi.
% 	\item Perintah \code{\bslash{}overline\{\}} untuk menulis garis di atas suatu ekspresi.
% 	\item Perintah \code{\bslash{}text\{\}} untuk menulis teks biasa di dalam persamaan.
% 	\item Di dalam teks biasa, kita bisa menuliskan kembali sebuah persamaan matematis dengan menambahkan tanda \code{\$} di awal dan di akhir persamaan tersebut.
% 		Di \gls{latex}, terdapat dua mode \f{typesetting}, yaitu mode teks dan mode matematika.
% 		Fungsi \code{\$} adalah untuk mengubah mode teks menjadi mode matematika.
% 	\item Perintah \code{\bslash{}vert} untuk menulis tanda garis vertikal.
% \end{itemize}

% \lstinputlisting[language={[latex]tex}, caption=Kode pembuatan \equ~\ref{equ:bola}, label=code:singleEquationBola]{assets/codes/tutorial/2-singleEquationBola.tex}

% Suatu persamaan yang dibuat menggunakan \f{environment} \code{align} akan secara otomatis memiliki indeks (nomor) dari persamaan.
% Perintah \code{align} ini juga dapat digunakan untuk menulis lebih dari satu persamaan.


% %-----------------------------------------------------------------------------%
% \subsection{Lebih dari Satu Persamaan}
% \label{sec:multiEqu}
% %-----------------------------------------------------------------------------%
% \equ~\ref{equ:matriks} adalah contoh persamaan matriks yang dibuat menggunakan \gls{latex}.

% \noindent \begin{align}\label{equ:matriks}
	|\overline{a} * \overline{b}| &= |\overline{a}| |\overline{b}| \sin\theta
	\\[0.2cm]
	\overline{a} * \overline{b} &=
	\begin{array}{| c c c |}
		\hat{i} & x_{1} & x_{2} \\
		\hat{j} & y_{1} & y_{2} \\
		\hat{k} & z_{1} & z_{2} \\
	\end{array} \nonumber \\[0.2cm]
	&= \hat{i} \,
	\begin{array}{ | c c | }
		y_{1} & y_{2} \\
		z_{1} & z_{2} \\
	\end{array}
	+ \hat{j} \,
	\begin{array}{ | c c | }
		z_{1} & z_{2} \\
		x_{1} & x_{2} \\
	\end{array}
	+ \hat{k} \,
	\begin{array}{ | c c | }
		x_{1} & x_{2} \\
		y_{1} & y_{2} \\
	\end{array}
	\nonumber
\end{align}

\addequtotoc{equ:matriks}{Persamaan matriks}


% Pada \equ~\ref{equ:matriks} dapat dilihat beberapa baris persamaan menjadi satu bagian dari persamaan tersebut.
% Kode untuk menyusun kumpulan persamaan di \equ~\ref{equ:matriks} dapat dilihat pada \lst~\ref{code:multiEquationAsOne}.
% Eksekusi perintah \code{\bslash{}addequtotoc\{label\}\{caption\}} juga cukup dilakukan sekali saja.

% \lstinputlisting[language={[latex]tex}, caption=Kode pembuatan \equ~\ref{equ:matriks}, label=code:multiEquationAsOne]{assets/codes/tutorial/2-multiEquationAsOne.tex}

% Terdapat beberapa perintah yang digunakan untuk menyusun \equ~\ref{equ:matriks}, yaitu:
% \begin{itemize}
% 	\item Perintah \code{\bslash{}begin\{array\}} dan \code{\bslash{}end\{array\}} untuk membuat matriks.
% 	Cara kerja \f{environment} \code{array} sama dengan \f{environment} \code{tabular} yang digunakan untuk membuat tabel.
% 	\item Perintah \code{\bslash{}sin} untuk menulis fungsi sinus.
% 	\item Perintah \code{\bslash{}theta} untuk menulis simbol theta ($\theta$).
% 	\item Perintah \code{\bslash{}hat\{\}} untuk menulis tanda aksen \f{hat} (\^{}) di atas suatu huruf.
% \end{itemize}

% Sedangkan dibawah ini dapat dilihat bahwa dengan cara yang sama, \equ~
% \ref{equ:integral}, \ref{equ:limit}, dan \ref{equ:eksponen} memiliki nomor
% persamaannya masing-masing.

% \noindent \begin{align}
	\label{equ:integral}
	\int_{a}^{b} f(x)\, dx + \int_{b}^{c} f(x) \, dx = \int_{a}^{c} f(x) \, dx \\
	\label{equ:limit}
	\lim_{x \to \infty} \frac{f(x)}{g(x)} = 0 \hspace{1cm}
	\text{jika pangkat $f(x)$ $<$ pangkat $g(x)$} \\
	\label{equ:eksponen}
	a^{m^{a \, ^{n}\log b }} = b^{\frac{m}{n}}
\end{align}

\addequtotoc{equ:integral}{Persamaan integral}
\addequtotoc{equ:limit}{Persamaan limit}
\addequtotoc{equ:eksponen}{Persamaan eksponen dan logaritma}


% Kode yang menyusun \equ~\ref{equ:integral}, \ref{equ:limit}, dan \ref{equ:eksponen} dapat dilihat pada \lst~\ref{code:multiEquationAsSeparate}.
% Perbedaan penyusunan tiga argumen tersebut dibandingkan dengan \equ~\ref{equ:matriks} adalah penggunaan \f{label} pada setiap persamaan.
% Pada \equ~\ref{equ:matriks}, \f{label} diletakkan pada bagian awal persamaan pertama, sehingga hanya satu \f{label} yang digunakan.
% Pada \equ~\ref{equ:integral}, \ref{equ:limit}, dan \ref{equ:eksponen}, \f{label} diletakkan pada awal setiap persamaan, sehingga setiap persamaan memiliki \f{label} masing-masing.
% Selain itu, jika ingin memasukkan setiap persamaan secara terpisah ke Daftar Persamaan,
% eksekusi perintah \code{\bslash{}addequtotoc\{label\}\{caption\}} juga harus dilakukan terpisah untuk setiap persamaan.

% \lstinputlisting[language={[latex]tex}, caption={Kode pembuatan \equ~\ref{equ:integral}, \ref{equ:limit}, dan \ref{equ:eksponen}}, label=code:multiEquationAsSeparate]{assets/codes/tutorial/2-multiEquationAsSeparate.tex}

% Terdapat beberapa perintah yang digunakan untuk menyusun \equ~\ref{equ:integral}, \ref{equ:limit}, dan \ref{equ:eksponen}, yaitu:
% \begin{itemize}
% 	\item Perintah \code{\bslash{}int} untuk menulis simbol integral.
% 	Untuk batas bawah dan batas atas integral, bisa dituliskan dengan menggunakan perintah \f{subscript} dan \f{superscript}.
% 	\item Perintah \code{\bslash{}lim} untuk menulis simbol limit.
% 	\item Perintah \code{\bslash{}to} untuk menulis panah ke kanan.
% 	\item Perintah \code{\bslash{}infty} untuk menulis simbol tak hingga.
% 	\item Perintah \code{\bslash{}log} untuk menulis fungsi logaritma.
% 	Pangkat basis logaritma dituliskan dengan menggunakan perintah \f{superscript} sebelum perintah \code{\bslash{}log}.
% \end{itemize}


% %-----------------------------------------------------------------------------%
% \section{Menambahkan Kode Program}
% \label{sec:codeListing}
% % Hal baru di template 2017
% %-----------------------------------------------------------------------------%
% % v2.2.1: tutorial dipindah dari Bab 3 ke Bab 2
% Pada \gls{latex}, kode program seringkali disebut \f{listing}.
% \f{Syntax highlighting} kini sudah bisa dilakukan secara otomatis oleh \f{library} yang ada di \gls{latex}.
% Sudah tidak perlu lagi membuat skrip manual untuk menambahkan \f{syntax highlighting} sendiri.
% \lst~\ref{code:java} adalah contoh kode program (\f{listing}) Java yang dicetak oleh \gls{latex}.

% \lstinputlisting[language=Java, caption=Kode sampel Java yang cukup panjang, label=code:java]{assets/codes/2-sample.java}

% Sintaks untuk memasukkan kode program ke dalam dokumen \gls{latex}~adalah sebagai berikut:

% \begin{lstlisting}[language={[latex]tex}, caption=Meng]
% \lstinputlisting[language=Java, caption=Kode sampel Java yang cukup panjang, label=code:java]{assets/codes/2-sample.java}
% \end{lstlisting}

% Terdapat tiga argumen yang digunakan pada perintah \code{\bslash{}lstinputlisting}:
% \begin{itemize}
% 	\item \code{language} digunakan untuk menentukan bahasa pemrograman yang digunakan.
% 	Untuk menggunakan suatu dialek bahasa pemrograman yang berbeda dari \f{default},
% 	misalkan versi Python3 dari Python,
% 	gunakan perintah \code{language=\{[3]Python\}}.
% 	\item \code{caption} digunakan untuk memberikan \f{caption} pada kode program.
% 	Argumen ini sifatnya opsional, jika ada, maka \f{caption} akan ditampilkan di bawah kode program.
% 	Jika argumen ini tidak ada, maka \f{caption} tidak akan ditampilkan dan kode tidak bisa masuk ke daftar kode program.
% 	\item \code{label} digunakan untuk memberikan label pada kode program untuk rujukan di dalam dokumen (\f{cross-reference}).
% 	Argumen ini tidak boleh didefinisikan jika argumen \code{caption} tidak didefinisikan.
% \end{itemize}

% Terdapat empat kelompok bahasa pemrograman (dan dialek) yang didukung oleh implementasi \code{listings} pada \f{template} ini, yaitu:

% \begin{itemize}
% 	\item \bo{Bahasa pemrograman yang didukung secara \f{default} oleh \code{listings}} (menurut \cite{latex:source_code_listings}): \\
% 		\code{ABAP}, \code{ACSL}, \code{Ada}, \code{Algol}, \code{Ant}, \code{Awk}, \code{bash}, \code{Basic}, \code{C++}, \code{C}, \code{Caml},
% 		\code{Clean}, \code{Cobol}, \code{Comal}, \code{command.com} (Windows Batch), \code{csh}, \code{Delphi}, \code{Eiffel}, \code{Elan},
% 		\code{erlang}, \code{Euphoria}, \code{Fortran}, \code{GCL}, \code{Go} (golang), \code{Gnuplot}, \code{Haskell}, \code{HTML}, \code{IDL},
% 		\code{inform}, \code{Java}, \code{JVMIS}, \code{ksh}, \code{Lisp}, \code{Logo}, \code{Lua}, \code{make}, \code{Mathematica}, \code{Matlab},
% 		\code{Mercury}, \code{MetaPost}, \code{Miranda}, \code{Mizar}, \code{ML}, \code{Modelica}, \code{Modula-2}, \code{MuPAD}, \code{NASTRAN},
% 		\code{Oberon-2}, \code{OCL}, \code{Octave}, \code{Oz}, \code{Pascal}, \code{Perl}, \code{PHP}, \code{PL/I}, \code{Plasm}, \code{POV},
% 		\code{Prolog}, \code{Promela}, \code{PSTricks}, \code{Python}, \code{R}, \code{Reduce}, \code{Rexx}, \code{RSL}, \code{Ruby}, \code{S},
% 		\code{SAS}, \code{Scilab}, \code{sh}, \code{SHELXL}, \code{Simula}, \code{SQL}, \code{tcl}, \code{TeX}, \code{VBScript}, \code{Verilog},
% 		\code{VHDL}, \code{VRML}, \code{XML}, dan \code{XSLT}.
% 	\item \bo{Dialek yang didukung secara \f{default} oleh \code{listings}} (menurut \cite{latex:source_code_listings}, diambil beberapa contoh):
% 		\begin{itemize}
% 			\item Dialek Assembly: \code{[Motorola68k]\{Assembler\}}, \code{[x86masm]\{Assembler\}},
% 			\item Dialek Awk: \code{[gnu]\{Awk\}} (GNU Awk), \code{[POSIX]\{Awk\}},
% 			\item Dialek C: \code{[ANSI]\{C\}} (default), \code{[Handel]\{C\}}, \code{[Objective]\{C\}} (Objective-C), \code{[Sharp]\{C\}} (C\#),
% 			\item Dialek Caml: \code{[Objective]\{Caml\}} (OCaml), \code{[light]\{Caml\}} (default),
% 			\item Dialek C++: \code{[11]\{C++\}}, \code{[ANSI]\{C++\}}, \code{[GNU]\{C++\}}, \code{[Visual]\{C++\}} (Visual C++),
% 				\code{[ISO]\{C++\}} (default),
% 			\item Dialek Java: \code{[]\{Java\}} (default), \code{[AspectJ]\{Java\}},
% 			\item Dialek Pascal: \code{[Borland6]\{Pascal\}}, \code{[XSC]\{Pascal\}}, \code{[Standard]\{Pascal\}} (default),
% 			\item Dialek Python: \code{[2]\{Python\}} (default, Python 2), \code{[3]\{Python\}} (Python 3),
% 			\item Dialek TeX: \code{[LaTeX]\{TeX\}} (LaTeX), \code{[AlLaTeX]\{TeX\}}, \code{[plain]\{TeX\}} (plain TeX, default),
% 			\item Dialek tcl: \code{[]\{Tcl\}} (default Tcl), \code{[tk]\{Tcl\}} (Tcl/Tk).
% 		\end{itemize}
% 	\item \bo{Bahasa pemrograman yang ditambahkan pada \f{template} ini}: \\
% 		\code{ABS}, \code{Acceleo}, \code{Batch}, \code{Clojure}, \code{CSS}, \code{D}, \code{Dart}, \code{Docker}, \code{F\#} (FSharp),
% 		\code{GDScript} (Godot), \code{GLSL}, \code{Groovy}, \code{HLSL}, \code{JavaScript}, \code{Julia}, \code{Kotlin}, \code{Markdown},
% 		\code{PowerShell}, \code{Rust}, \code{Scala}, \code{Scheme}, \code{Solidity}, \code{Swift}, \code{TOML}, \code{TypeScript}, dan \code{YAML}.
% 	\item \bo{Dialek yang ditambahkan pada \f{template} ini}:
% 		\begin{itemize}
% 			\item Dialek HTML: \code{[v5]\{HTML\}} (HTML5),
% 			\item Dialek Java: \code{[v9]\{Java\}} (Java 9 Modules), \code{[ContextJ]\{Java\}}, \code{[DeltaJ]\{Java\}}, dan \code{[FOP]\{Java\}}.
% 		\end{itemize}
% \end{itemize}

% Berikut adalah contoh penggunaan dialek bahasa pemrograman dalam menuliskan kode program.
% Dalam kasus ini, \lst~\ref{code:python2} merupakan kode Python versi 2.
% Sedangkan, \lst~\ref{code:python3} merupakan kode Python versi 3.
% Perbedaan antara kedua versi tersebut cukup signifikan, salah satunya adalah pada perintah \code{print}.
% Pada Python 2, perintah \code{print} tidak memerlukan tanda kurung karena merupakan sebuah \f{keyword},
% sedangkan pada Python 3, perintah \code{print} memerlukan tanda kurung karena merupakan sebuah \f{function}.

% \lstinputlisting[language=Python, caption=Kode Python 2, label=code:python2]{assets/codes/2-sample-python2.py}

% \lstinputlisting[language={[3]Python}, caption=Kode Python 3, label=code:python3]{assets/codes/2-sample-python3.py}

% Kode yang mencetak \lst~\ref{code:python2} dan \lst~\ref{code:python3} adalah sebagai berikut:

% \begin{lstlisting}[language={[latex]tex}]
% \lstinputlisting[language=Python, caption=Kode Python 2, label=code:python2]{assets/codes/2-sample-python2.py}
% \lstinputlisting[language={[3]Python}, caption=Kode Python 3, label=code:python3]{assets/codes/2-sample-python3.py}
% \end{lstlisting}

% Secara \f{default}, dialek yang digunakan untuk Python pada \f{library} \code{listings} adalah Python 2.
% Sehingga untuk mencetak kode Python 3, perlu digunakan dialek Python 3 (\code{\{[3]Python\}}).

% Kode program yang dicetak oleh \gls{latex} bersifat \f{auto-wrapped}.
% Jika suatu baris kode program melebihi batas lebar halaman,
% maka kode program tersebut akan dipindahkan ke baris berikutnya.
% \f{Auto-wrapped} ini berguna agar Anda tidak perlu memberikan \f{line break} manual pada kode Anda,
% Anda bisa menyampaikan kode program Anda apa adanya.

% \bo{Catatan}: Jangan lupa untuk menjelaskan kode melalui paragraf,
% terutama pada bagian-bagian yang perlu penjelasan lebih.
% Setiap kode perlu dijelaskan, kalau bisa rujuk ke setiap baris, karena belum tentu pembaca mau membaca kode Anda.
% Tetapi, pembaca tetap perlu mengetahui ide di balik kode yang Anda buat, dan mengapa kode tersebut dibuat.



% %-----------------------------------------------------------------------------%
% \section{Keterkaitan Teori Dengan Penelitian}
% \label{sec:keterkaitan}
% %-----------------------------------------------------------------------------%
% \todo{
% 	Ada baiknya setelah menjelaskan teori-teori, Anda menjelaskan apa kaitan teori tersebut dengan penelitian Anda.
% 	Hal ini tentunya membantu pembaca dalam memahami bahwa teori yang Anda paparkan memang penting untuk memahami penelitian Anda nantinya.
% }

% \begin{figure}
% 	\centering
% 	\includegraphics[width=\textwidth]{assets/pics/research_concept_map.png}
% 	\caption{Keterkaitan konsep hasil studi literatur terhadap penelitian}
% 	\label{fig:research_concept_map}
% \end{figure}

% \todo{
% 	Jelaskan \pic~\ref{fig:research_concept_map} di sini.
% 	Setiap gambar pada tugas akhir butuh penjelasan.
% 	Gambar hadir untuk mempermudah membaca memahami konteks, tetapi tidak bisa berdiri sendiri tanpa penjelasan.
% }
