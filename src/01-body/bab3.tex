%-----------------------------------------------------------------------------%
\chapter{\babTiga}
\label{bab:3}
%-----------------------------------------------------------------------------%
Bab ini menjelaskan tahapan yang dilakukan selama proses penelitian.
% Hal ini mencakup bagaimana cara menulis persamaan matematis di \gls{latex}, menambahkan daftar isi, catatan, \acrshort{pdf}, menambahkan kode, bahkan menambahkan perintah baru.

% \todo{
% 	Sejatinya bab ini digunakan untuk membahas inti dari penelitian Anda.
% 	Sesuaikan saja dengan kebutuhkan Anda: misalkan bab tiga Anda adalah penjelasan terkait desain sistem.
% }


%-----------------------------------------------------------------------------%
\section{Pendekatan Penelitian}
\label{sec:pendekatanPenelitian}
%-----------------------------------------------------------------------------%
Penelitian ini menggunakan pendekatan \textbf{kuantitatif dan eksperimental}. Tujuan utama dari penelitian ini adalah untuk membandingkan kinerja berbagai algoritma fundamental ketika diimplementasikan dalam bahasa pemrograman yang berbeda, yaitu Java, Python, PHP, dan Rust. Pengujian dilakukan dengan mengimplementasikan setiap algoritma dalam keempat bahasa tersebut, lalu mengukur performa eksekusinya berdasarkan metrik waktu proses (\textit{execution time}) dan penggunaan memori (\textit{memory usage}).


\section{Tujuan Eksperimen}
\label{sec:tujuanEksperimen}
Tujuan dari eksperimen ini adalah untuk:
\begin{itemize}
	\item perbandingan performa waktu dan memori antara bahasa pemrograman dalam menjalankan algoritma tertentu
	\item kelebihan dan kekurangan dari masing-masing bahasa pemrograman dalam konteks algoritma yang diuji
	\item Mengetahui bagaimana performa algoritma berubah tergantung bahasa pemrograman yang digunakan.
	\item Mengamati dampak perbedaan paradigma, interpreter/kompiler, dan sistem manajemen memori dari masing-masing bahasa terhadap eksekusi algoritma.
	\item Memberikan rekomendasi dan insight terkait pemilihan bahasa dalam konteks kebutuhan algoritmik tertentu.
\end{itemize}

\section{Algoritma yang Diuji}
\label{sec:algoritmaDiuji}
\begin{enumerate}
	% \item sorting (bubble sort)
	% \item perkalian matrix (naive matrix multiplication)
	% \item integral numerik
	% \item graph traversal (DFS, BFS)
	% \item knapsack problem
	% \item string maching
	% \item path-finding algorithm
	% \item huffman coding
	\item \textbf{Bubble Sort} - Algoritma pengurutan sederhana.
	\item \textbf{Matrix Multiplication (Naive)} - Operasi numerik pada matriks.
	\item \textbf{Trapezoidal Rule} - Pendekatan numerik untuk menghitung integral.
	\item \textbf{Depth-First Search (DFS)} dan \textbf{Breadth-First Search (BFS)} - Algoritma penelusuran graf.
	\item \textbf{0/1 Knapsack dan Coin Change} - Algoritma \textit{dynamic programming}.
	\item \textbf{Naive String Matching} - Pencocokan pola sederhana pada string.
	\item \textbf{Dijkstra’s Algorithm} - Algoritma pencarian jalur terpendek.
	\item \textbf{Huffman Coding} - Algoritma kompresi berbasis pohon biner.
\end{enumerate}

Setiap algoritma dipilih karena mewakili kategori yang berbeda dan relevan untuk uji performa lintas bahasa pemrograman.

\section{Bahasa Pemrograman}
Empat bahasa pemrograman yang digunakan dalam penelitian ini adalah:
\begin{itemize}
	% \item Java
	% \item Python
	% \item Rust
	% \item PHP
	\item \textbf{Java} - Bahasa pemrograman berorientasi objek dengan tipe statis dan dikompilasi ke dalam bytecode.
	\item \textbf{Python} - Bahasa scripting tingkat tinggi yang dinamis dan diinterpretasi.
	\item \textbf{PHP} - Bahasa scripting server-side yang umum digunakan untuk pengembangan web.
	\item \textbf{Rust} - Bahasa pemrograman sistem yang modern, dikompilasi, dan berorientasi pada keamanan memori.
\end{itemize}

Pemilihan keempat bahasa ini bertujuan untuk membandingkan berbagai paradigma (compiled vs interpreted, static vs dynamic typing) serta keunggulan dan kelemahan masing-masing dalam konteks implementasi algoritma.

\section{Tools dan Lingkungan Pengujian}
\label{sec:perangkatPengujian}
Eksperimen dilakukan dalam lingkungan yang terkontrol, dengan spesifikasi sebagai berikut:
\begin{itemize}
	\item perangkat keras \\
	      \todo{
		      Harusnya GCP \\
		      Segera minta Pak Ari !!!
	      }
	\item perangkat lunak:
	      \begin{itemize}
		      \item bahasa dan versi
		      \item editor dan alat bantu
	      \end{itemize}
\end{itemize}

\section{Skenario Pengujian}
\label{sec:skenarioPengujian}
Pembuatan \f{testcase} dilakukan dengan menggunakan \textit{random generator}. Pengukuran akan dilakukan terhadap:
\begin{itemize}
	\item waktu eksekusi
	\item penggunaan memori
\end{itemize}

\section{Pengumpulan dan Analisis Data}
\label{sec:pengumpulanAnalisisData}
Data yang diperoleh dari hasil eksekusi akan dicatat dalam tabel. Selanjutnya akan divisualisasikan dalam bentuk grafik untuk memudahkan analisis. Data dibandingkan untuk melihat tren performa antar bahasa.

Langkah - langkah analisis:
\begin{enumerate}
	\item Menghitung rata-rata waktu eksekusi dari setiap algoritma dalam setiap bahasa.
	\item Menyajikan hasil dalam bentuk tabel dan grafik (\textit{bar chart / line chart}).
	\item Melakukan interpretasi hasil, khususnya jika ada hasil yang tidak konsisten atau menunjukkan perbedaan signifikan.
\end{enumerate}