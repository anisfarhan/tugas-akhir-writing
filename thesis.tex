%
% Template Laporan Skripsi/Thesis Universitas Indonesia
%
% @author  Ichlasul Affan, Azhar Kurnia
% @version 2.2.0
%
% Dokumen ini dibuat berdasarkan standar IEEE dalam membuat class untuk
% LaTeX dan konfigurasi LaTeX yang digunakan Fahrurrozi Rahman ketika
% membuat laporan skripsi, yang kemudian diadaptasi oleh Andreas Febrian dan
% Lia Sadita untuk template skripsi tahun 2010.
% Konfigurasi template sebelumnya telah disesuaikan dengan
% aturan penulisan thesis yang dikeluarkan UI pada tahun 2017.
%

%
% Tipe dokumen adalah report dengan satu kolom.
%
\documentclass[12pt, a4paper, onecolumn, twoside, final]{report}
\raggedbottom

\usepackage[none]{hyphenat}
% Load konfigurasi LaTeX untuk tipe laporan thesis
\usepackage{_internals/uithesis}
%
\UseRawInputEncoding
% Load konfigurasi khusus untuk laporan yang sedang dibuat
\input{config/settings}
% Daftar pemenggalan suku kata dan istilah dalam LaTeX
\include{_internals/hypeindonesia}
% Daftar istilah yang mungkin perlu ditandai
\input{config/istilah}

% Awal bagian penulisan laporan
\begin{document}

%
% Sampul Laporan
%
\include{_internals/sampul}
\ifodd\thechapterpagecount\forceclearchapter\fi

%
% Gunakan penomoran romawi untuk nomor halaman
%
\pagenumbering{roman}
%

%
% load halaman judul dalam
\strcompare{Kampus Merdeka}{\type}{} {
	\addtocontents{toc}{\protect\addvspace{-10pt}}
	\addChapter{Halaman Judul}
	\include{_internals/judul_dalam}
	\ifodd\thechapterpagecount\forceclearchapter\fi
}

%
% load halaman orisinalitas

\strcompare{Laporan Kerja Praktik}{\type}{} {
\strcompare{Kampus Merdeka}{\type}{} {
	\addtocontents{toc}{\protect\addvspace{-10pt}}
    \addChapter{Halaman Orisinalitas}
	\include{src/00-frontMatter/pernyataanOrisinalitas}
	\ifodd\thechapterpagecount\forceclearchapter\fi
}}

% Memunculkan penomoran romawi untuk halaman-halaman awal
\pagenumbering{roman}


% setelah bagian ini, halaman dihitung sebagai halaman ke 2 atau 3
\strcompare{Laporan Kerja Praktik}{\type}{\setcounter{page}{2}} {
	\strcompare{Kampus Merdeka}{\type}{\setcounter{page}{2}} {
		\setcounter{page}{3}
	}
}

%
% Lembar Penegesahan
\strcompare{Laporan Kerja Praktik}{\type}
{
	% Lembar Pengesahan Kerja Praktik dari LaTeX
	\addChapter{Lembar Persetujuan Dosen Kerja Praktik}
	\include{src/00-frontMatter/pengesahanKP}
	\ifodd\thechapterpagecount\forceclearchapter\fi
}{
\strcompare{Kampus Merdeka}{\type}
{
	\addChapter{Lembar Pengesahan}
	\include{src/00-frontMatter/pengesahanMBKM}
	\ifodd\thechapterpagecount\forceclearchapter\fi
}
{
	\addChapter{Lembar Pengesahan}
	% Gunakan salah satu (comment atau hapus kode yang tidak perlu):
	% Lembar Pengesahan Tugas Akhir dari LaTeX
	\strcompare{Doktor}{\jenjang}
	{\include{src/00-frontMatter/pengesahanSidangS3}}
	{\include{src/00-frontMatter/pengesahanSidang}}
	% Lembar Pengesahan dari PDF lain (misal: generated oleh SISIDANG [Fasilkom])
	%\putpdf{assets/pdfs/pengesahanSidang.pdf}
	\ifodd\thechapterpagecount\forceclearchapter\fi
}}


\strcompare{Laporan Kerja Praktik}{\type}{} {
\strcompare{Kampus Merdeka}{\type}{} {
	%
	% Kata Pengantar
	%
	\addChapter{\kataPengantar}
	\include{src/00-frontMatter/kataPengantar}
	\ifodd\thechapterpagecount\forceclearchapter\fi

	%
	% Lembar Persetujuan Publikasi Ilmiah
	%
	\addChapter{Lembar Persetujuan Karya Ilmiah}
	\include{src/00-frontMatter/persetujuanPublikasi}
	\ifodd\thechapterpagecount\forceclearchapter\fi
}}

%
% Untuk halaman pertama setiap chapter mulai dari abstrak, tetap berikan mark universitas.
%
\pagestyle{first-pages}

%
\addChapter{Abstrak}
\include{src/00-frontMatter/abstrak}

\include{src/00-frontMatter/abstract}

%
% Daftar Isi, Gambar, dan Tabel
%
\addDefaultListPage{\tableofcontents}
\addDefaultListPage{\listoffigures}
\addDefaultListPage{\listoftables}

%
% Daftar Kode Program
% Comment to disable.
%
\addCustomListPage{\listoflistings}{\lstlistlistingname}

%
% Daftar Equation (Persamaan Matematis)
% Uncomment to use.
%
% \addCustomListPage{\listofequ}{\listofequname}

%
% Daftar Isi yang Didefinisikan Sendiri (Custom)
% Definisi jenis objek baru dapat dilakukan di uithesis.sty
% Uncomment to use.
%
% \addCustomListPage{\listofthing}{\listofthingname}

%
% Daftar Lampiran
% Comment to disable.
%
\addCustomListPage{\listofappendix}{\listofappendixname}


\noCAPinToC % Revert to original \addcontentsline formatting
\ifodd\thechapterpagecount\clearpage\else\forceclearchapter\fi


% Jika penomoran romawi selesai di ganjil
%\naiveoddclearchapter
% Jika penomoran romawi selesai di genap
%\naiveevenclearchapter

%
% Gunakan penomoran Arab (1, 2, 3, ...) setelah bagian ini.
%
\pagenumbering{arabic}
\pagestyle{standard}

\setoddevenheader
%-----------------------------------------------------------------------------%
\chapter{\babSatu}
\label{bab:1}
%-----------------------------------------------------------------------------%
Pada bab ini, akan dijelaskan tentang latar belakang dan permasalahan yang diselesaikan pada penelitian ini.


%-----------------------------------------------------------------------------%
\section{Latar Belakang}
\label{sec:latarBelakang}
%-----------------------------------------------------------------------------%
% \todo{Tentukan latar belakang dari penelitian Anda di sini (\f{background}).}
\begin{itemize}
	\item Pentingnya algoritma dalam pengembangan perangkat lunak.
	\item Pengaruh bahasa pemrograman terhadap performa algoritma.
	\item Masalah umum: bahasa berbeda = hasil performa berbeda.
	\item Relevansi untuk dunia industri \& akademik.
\end{itemize}
\section{Rumusan Masalah}
\label{sec:rumusanMasalah}
%-----------------------------------------------------------------------------%
% \todo{Sebutkan permasalahan penelitian Anda dari latar belakang tersebut.}

\begin{itemize}
	\item Bagaimana performa beberapa algoritma klasik berbeda saat diimplementasikan pada bahasa pemrograman yang berbeda?
	\item Apa saja faktor yang mempengaruhi performa algoritma dalam konteks pengembangan perangkat lunak?
\end{itemize}

%-----------------------------------------------------------------------------%
\section{Tujuan Penelitian}
\label{sec:tujuanPenelitian}
%-----------------------------------------------------------------------------%

\begin{itemize}
	\item Menganalisis dan membandingkan performa algoritma dalam bahasa pemrograman yang berbeda
\end{itemize}

%-----------------------------------------------------------------------------%
\section{Batasan Masalah}
\label{sec:batasanMasalah}
%-----------------------------------------------------------------------------%
% Berikut ini adalah asumsi yang digunakan sebagai batasan penelitian ini:
% \begin{enumerate}
% 	\item Salah satu batasannya adalah, ini hanya \f{template}.
% \end{enumerate}

% \todo{Umumnya ada asumsi atau batasan yang digunakan untuk menjawab pertanyaan-pertanyaan penelitian diatas.}
\begin{itemize}
	\item Algoritma yang diuji: Sorting, Matrix Multiplication, BFS/DFS, Dijkstra, dll (8 total)
	\item Bahasa pemrograman: Java, Python, PHP, Rust
	\item Metrik evaluasi: waktu eksekusi, konsumsi memori (opsional: CPU usage)
	\item Eksperimen dilakukan dalam lingkungan sistem yang dikontrol
\end{itemize}


%-----------------------------------------------------------------------------%
% \section{Tujuan Penelitian}
% \label{sec:tujuan}
% %-----------------------------------------------------------------------------%
% Berikut ini adalah tujuan penelitian yang dilakukan:
% \begin{enumerate}
% 	\item Untuk memberikan \f{template} yang dapat mempermudah skripsi orang lain.
% \end{enumerate}

% \todo{Tuliskan tujuan penelitian Anda di bagian ini.}


%-----------------------------------------------------------------------------%
\section{Manfaat Penelitian}
\label{sec:manfaat}
%-----------------------------------------------------------------------------%
% Berikut ini adalah manfaat penelitian yang dilakukan:
% \begin{enumerate}
% 	\item Penelitian ini memberikan \f{template} yang harapannya dapat mempermudah skripsi orang lain.
% \end{enumerate}

% \todo{Tuliskan tujuan penelitian Anda di bagian ini.}

\begin{itemize}
	\item Memberikan referensi bagi developer/peneliti terkait pilihan bahasa dalam implementasi algoritma
	\item Menjadi acuan untuk pembelajaran pemrograman dan optimasi performa
\end{itemize}

%-----------------------------------------------------------------------------%
% \section{Posisi Penelitian}
% \label{sec:posisiPenelitian}
% %-----------------------------------------------------------------------------%
% \todo{
% 	Sebutkan posisi penelitian Anda. Ada baiknya jika Anda menggunakan gambar atau diagram.
% 	Template ini telah menyediakan contoh cara memasukkan gambar.
% }

% \begin{figure}
% 	\centering
% 	\includegraphics[width=0.4\textwidth]{assets/pics/makara.png}
% 	\caption{Penjelasan singkat terkait gambar.}
% 	\label{fig:research_position}
% \end{figure}

% \todo{
% 	Jelaskan \pic~\ref{fig:research_position} di sini.
% 	Setiap gambar yang dimasukkan ke tugas akhir \bo{WAJIB} untuk dijelaskan oleh minimal satu paragraf.
% }


%-----------------------------------------------------------------------------%
\section{Metodologi Penelitian}
\label{sec:metodologiPenelitian}
%-----------------------------------------------------------------------------%
% \todo{
% 	Subbab ini umumnya menjelaskan metode penelitian, jika metode yang digunakan cukup sederhana.
% 	Jika metode yang digunakan cukup kompleks atau ada kewajiban dari fakultas Anda untuk menjelaskan metode secara rinci, gunakan Bab 3 untuk menjelaskan Metodologi Penelitian.
% }

\begin{itemize}
	\item Implementasi algoritma di 4 bahasa
	\item Uji performa pada input beragam ukuran
	\item Bandingkan hasil berdasarkan metrik tertentu
\end{itemize}

% Berikut ini adalah langkah penelitian yang telah dilakukan:
% \begin{enumerate}
% 	\item Tinjauan literatur \\
% 	Pada tahap ini, dipelajari teori-teori yang terkait dengan penelitian ini untuk mendapatkan konsep dasar yang dibutuhkan dalam mencapai tujuan penelitian.
% 	\item Analisis implementasi dan kesimpulan \\
% 	Pada tahap ini, digunakan studi kasus untuk analisis terkait kegunaan \f{template}.
% 	Setelah melakukan analisis tersebut, ditarik kesimpulan keseluruhan dari penelitian ini.
% \end{enumerate}


%-----------------------------------------------------------------------------%
\section{Sistematika Penulisan}
\label{sec:sistematikaPenulisan}
%-----------------------------------------------------------------------------%
% Sistematika penulisan laporan adalah sebagai berikut:
% \begin{itemize}
% 	\item Bab 1 \babSatu \\
% 	    Bab ini mencakup latar belakang, cakupan penelitian, dan pendefinisian masalah.
% 	\item Bab 2 \babDua \\
% 	    Bab ini mencakup pemaparan terminologi dan teori yang terkait dengan penelitian berdasarkan hasil tinjauan pustaka yang telah digunakan, sekaligus memperlihatkan kaitan teori dengan penelitian.
% 	\item Bab 3 \babTiga \\
% 	    Apa itu Bab 3?
% 	\item Bab 4 \babEmpat \\
% 		Apa itu Bab 4?
% 	\item Bab 5 \babLima \\
% 	    Apa itu Bab 5?
% 	\item Bab 6 \kesimpulan \\
% 	    Bab ini mencakup kesimpulan akhir penelitian dan saran untuk pengembangan berikutnya.
% \end{itemize}

\begin{itemize}
	\item Bab 1: Pendahuluan
	\item Bab 2: Tinjauan Pustaka / Studi literatur
	\item Bab 3: Metodologi Penelitian
	\item Bab 4: Implementasi
	\item Bab 5: Hasil \& Evaluasi
\end{itemize}

% \todo{Anda bisa mengubah atau menambahkan penjelasan singkat mengenai isi masing-masing bab. Setiap tugas akhir pasti ada yang berbeda pada bagian ini.}

\clearchapter
\include{src/01-body/bab2}
\clearchapter
%-----------------------------------------------------------------------------%
\chapter{\babTiga}
\label{bab:3}
%-----------------------------------------------------------------------------%
Bab ini menjelaskan tahapan yang dilakukan selama proses penelitian.
% Hal ini mencakup bagaimana cara menulis persamaan matematis di \gls{latex}, menambahkan daftar isi, catatan, \acrshort{pdf}, menambahkan kode, bahkan menambahkan perintah baru.

% \todo{
% 	Sejatinya bab ini digunakan untuk membahas inti dari penelitian Anda.
% 	Sesuaikan saja dengan kebutuhkan Anda: misalkan bab tiga Anda adalah penjelasan terkait desain sistem.
% }


%-----------------------------------------------------------------------------%
\section{Pendekatan Penelitian}
\label{sec:pendekatanPenelitian}
%-----------------------------------------------------------------------------%
Penelitian ini menggunakan pendekatan \textbf{kuantitatif dan eksperimental}. Tujuan utama dari penelitian ini adalah untuk membandingkan kinerja berbagai algoritma fundamental ketika diimplementasikan dalam bahasa pemrograman yang berbeda, yaitu Java, Python, PHP, dan Rust. Pengujian dilakukan dengan mengimplementasikan setiap algoritma dalam keempat bahasa tersebut, lalu mengukur performa eksekusinya berdasarkan metrik waktu proses (\textit{execution time}) dan penggunaan memori (\textit{memory usage}).


\section{Tujuan Eksperimen}
\label{sec:tujuanEksperimen}
Tujuan dari eksperimen ini adalah untuk:
\begin{itemize}
	\item perbandingan performa waktu dan memori antara bahasa pemrograman dalam menjalankan algoritma tertentu
	\item kelebihan dan kekurangan dari masing-masing bahasa pemrograman dalam konteks algoritma yang diuji
	\item Mengetahui bagaimana performa algoritma berubah tergantung bahasa pemrograman yang digunakan.
	\item Mengamati dampak perbedaan paradigma, interpreter/kompiler, dan sistem manajemen memori dari masing-masing bahasa terhadap eksekusi algoritma.
	\item Memberikan rekomendasi dan insight terkait pemilihan bahasa dalam konteks kebutuhan algoritmik tertentu.
\end{itemize}

\section{Algoritma yang Diuji}
\label{sec:algoritmaDiuji}
\begin{enumerate}
	% \item sorting (bubble sort)
	% \item perkalian matrix (naive matrix multiplication)
	% \item integral numerik
	% \item graph traversal (DFS, BFS)
	% \item knapsack problem
	% \item string maching
	% \item path-finding algorithm
	% \item huffman coding
	\item \textbf{Bubble Sort} - Algoritma pengurutan sederhana.
	\item \textbf{Matrix Multiplication (Naive)} - Operasi numerik pada matriks.
	\item \textbf{Trapezoidal Rule} - Pendekatan numerik untuk menghitung integral.
	\item \textbf{Depth-First Search (DFS)} dan \textbf{Breadth-First Search (BFS)} - Algoritma penelusuran graf.
	\item \textbf{0/1 Knapsack dan Coin Change} - Algoritma \textit{dynamic programming}.
	\item \textbf{Naive String Matching} - Pencocokan pola sederhana pada string.
	\item \textbf{Dijkstra’s Algorithm} - Algoritma pencarian jalur terpendek.
	\item \textbf{Huffman Coding} - Algoritma kompresi berbasis pohon biner.
\end{enumerate}

Setiap algoritma dipilih karena mewakili kategori yang berbeda dan relevan untuk uji performa lintas bahasa pemrograman.

\section{Bahasa Pemrograman}
Empat bahasa pemrograman yang digunakan dalam penelitian ini adalah:
\begin{itemize}
	% \item Java
	% \item Python
	% \item Rust
	% \item PHP
	\item \textbf{Java} - Bahasa pemrograman berorientasi objek dengan tipe statis dan dikompilasi ke dalam bytecode.
	\item \textbf{Python} - Bahasa scripting tingkat tinggi yang dinamis dan diinterpretasi.
	\item \textbf{PHP} - Bahasa scripting server-side yang umum digunakan untuk pengembangan web.
	\item \textbf{Rust} - Bahasa pemrograman sistem yang modern, dikompilasi, dan berorientasi pada keamanan memori.
\end{itemize}

Pemilihan keempat bahasa ini bertujuan untuk membandingkan berbagai paradigma (compiled vs interpreted, static vs dynamic typing) serta keunggulan dan kelemahan masing-masing dalam konteks implementasi algoritma.

\section{Tools dan Lingkungan Pengujian}
\label{sec:perangkatPengujian}
Eksperimen dilakukan dalam lingkungan yang terkontrol, dengan spesifikasi sebagai berikut:
\begin{itemize}
	\item perangkat keras \\
	      \todo{
		      Harusnya GCP \\
		      Segera minta Pak Ari !!!
	      }
	\item perangkat lunak:
	      \begin{itemize}
		      \item bahasa dan versi
		      \item editor dan alat bantu
	      \end{itemize}
\end{itemize}

\section{Skenario Pengujian}
\label{sec:skenarioPengujian}
Pembuatan \f{testcase} dilakukan dengan menggunakan \textit{random generator}. Pengukuran akan dilakukan terhadap:
\begin{itemize}
	\item waktu eksekusi
	\item penggunaan memori
\end{itemize}

\section{Pengumpulan dan Analisis Data}
\label{sec:pengumpulanAnalisisData}
Data yang diperoleh dari hasil eksekusi akan dicatat dalam tabel. Selanjutnya akan divisualisasikan dalam bentuk grafik untuk memudahkan analisis. Data dibandingkan untuk melihat tren performa antar bahasa.

Langkah - langkah analisis:
\begin{enumerate}
	\item Menghitung rata-rata waktu eksekusi dari setiap algoritma dalam setiap bahasa.
	\item Menyajikan hasil dalam bentuk tabel dan grafik (\textit{bar chart / line chart}).
	\item Melakukan interpretasi hasil, khususnya jika ada hasil yang tidak konsisten atau menunjukkan perbedaan signifikan.
\end{enumerate}
\clearchapter
%-----------------------------------------------------------------------------%
\chapter{\babEmpat}
\label{bab:4}
%-----------------------------------------------------------------------------%
Bab ini menjelaskan tentang implementasi dari kode-kode yang dipilih pada tugas akhir ini.

\todo{
    Sejatinya bab ini digunakan untuk membahas inti dari penelitian Anda.
    Sesuaikan saja dengan kebutuhkan Anda: misalkan bab empat Anda adalah penjelasan terkait implementasi sistem.
}


%-----------------------------------------------------------------------------%
\section{Sorting - Bubble Sort}
\label{sec:bubble-sort-implementation}
%-----------------------------------------------------------------------------%
\subsection{Deskripsi Algoritma}
Bubble Sort merupakan algoritma pengurutan sederhana yang efektif untuk dataset kecil. Algoritma ini bekerja dengan cara melakukan iterasi berulang pada larik data dan menukar pasangan elemen bersebelahan jika berada dalam urutan yang salah. Proses ini diulang hingga seluruh elemen berada dalam urutan yang benar. Kompleksitas waktu dalam kasus terburuk dan rata-rata adalah $\mathcal{O}\left(N^2\right)$, sedangkan kompleksitas ruang adalah ${O}\left(1\right)$.

\subsection{Pseudocode}
\begin{verbatim}
    procedure bubbleSort(arr, n):
        for i from 0 to n-1:
            for j from 0 to n-i-1:
                if arr[j] > arr[j+1]:
                    swap arr[j] and arr[j+1]
\end{verbatim}

\subsection{Contoh Input-Output}
\textbf{Input:}
\begin{verbatim}
    6
    3 2 1 5 6 4
\end{verbatim}

\textbf{Output:}
\begin{verbatim}
    1 2 3 4 5 6
\end{verbatim}

\section{Matrix Multiplication - Naive}
\label{sec:matrix-multiplication-implementation}

\subsection{Deskripsi Algoritma}
Matrix Multiplication adalah proses mengalikan dua matriks untuk menghasilkan matriks baru. Jika matriks A berukuran $m \times n$ dan matriks B berukuran $n \times p$, maka hasil kali matriks C = A * B akan berukuran $m \times p$. Proses ini melibatkan penjumlahan produk elemen baris dari matriks A dengan elemen kolom dari matriks B. Kompleksitas waktu untuk metode naive adalah $\mathcal{O}\left(N^3\right)$, sedangkan kompleksitas ruang adalah $\mathcal{O}\left(N^2\right)$.

\subsection{Pseudocode}
\begin{verbatim}
    function matrixMultiply(A, B):
        for i from 0 to A.rows-1:
            for j from 0 to B.cols-1:
                C[i][j] = 0
                for k from 0 to A.cols-1:
                    C[i][j] += A[i][k] * B[k][j]

        return C
\end{verbatim}

\subsection{Contoh Input-Output}
\textbf{Input:}
\begin{verbatim}
    matrix A size:
    2 3
    matrix A:
    1 2 3
    4 5 6

    matrix B size:
    3 2
    matrix B:
    1 1
    1 2
    2 3
\end{verbatim}

\textbf{Output:}
\begin{verbatim}
    result matrix:
    9 14
    21 32
\end{verbatim}

\section{Integral Modeling - Trapezoidal Rule Integration}
\label{sec:trapezoidal-rule-integration-implementation}

\subsection{Deskripsi Algoritma}
Trapezoidal Rule adalah metode numerik untuk menghitung integral dari fungsi. Metode ini bekerja dengan membagi area di bawah kurva menjadi beberapa trapezoid, kemudian menjumlahkan luas trapezoid tersebut untuk mendapatkan perkiraan nilai integral. Kompleksitas waktu dari metode ini adalah $\mathcal{O}(n)$, di mana $n$ adalah jumlah subinterval (dengan asumsi perhitungan nilai fungsi dilakukan dalam $\mathcal{O}(1)$), dan kompleksitas ruangnya adalah $\mathcal{O}(1)$.

\subsection{Pseudocode}
\begin{verbatim}
    function trapezoidalRule(f, a, b, n):
        h = (b - a) / n
        sum = f(a) + f(b)
        for i from 1 to n-1:
            sum += 2 * f(a + i * h)
        return sum * h / 2
\end{verbatim}

\subsection{Contoh Input-Output}
\textbf{Input:}
\begin{verbatim}
    f(x) = x^2
    a = 0
    b = 1
    n = 1000
\end{verbatim}

\textbf{Output:}
\begin{verbatim}
    0.333333
\end{verbatim}

\section{Graph Traversal - Depth-First Search (DFS)}
\label{sec:dfs-implementation}

\subsection{Deskripsi Algoritma}
Depth-First Search (DFS) adalah algoritma pencarian yang digunakan untuk menjelajahi graf atau pohon. DFS bekerja dengan cara menjelajahi sejauh mungkin di sepanjang cabang sebelum melakukan backtrack. Algoritma ini menggunakan struktur data stack (atau rekursi) untuk melacak jalur yang sedang dieksplorasi. Kompleksitas waktu DFS adalah $\mathcal{O}\left(V + E\right)$, di mana $V$ adalah jumlah vertex dan $E$ adalah jumlah edge. Kompleksitas ruangnya adalah $\mathcal{O}\left(V\right)$ untuk menyimpan status kunjungan.

\subsection{Pseudocode}
\begin{verbatim}
    procedure DFS(graph, start):
        create a stack S
        push start onto S
        mark start as visited

        while S is not empty:
            vertex = pop S
            for each neighbor of vertex:
                if neighbor is not visited:
                    mark neighbor as visited
                    push neighbor onto S
\end{verbatim}

\subsection{Contoh Input-Output}
Pada contoh ini algoritma DFS digunakan untuk menghitung luas pulau dari koordinat (x, y) yang diberikan. Dimana 1 menandakan daratan dan 0 menandakan lautan.
\textbf{Input:}
\begin{verbatim}
    grid size:
    5 5
    grid:
    1 1 0 0 0
    1 1 0 1 0
    0 0 0 1 1
    0 0 1 1 1
    1 1 0 0 0
    start point:
    1 1
\end{verbatim}

\textbf{Output:}
\begin{verbatim}
    area of island: 4
\end{verbatim}

\section{Dynamic Programming - Coin Change}
\label{sec:coin-change-implementation}

\subsection{Deskripsi Algoritma}
Contoh masalah algoritma dynamic programming diberikan dalam bentuk Coin Change Problem. Masalah ini bertujuan untuk menemukan jumlah minimum koin yang dibutuhkan untuk mencapai nilai tertentu dengan denominasi koin yang diberikan. Kompleksitas waktu adalah $\mathcal{O}\left(N \cdot M\right)$, di mana $N$ adalah jumlah nilai yang ingin dicapai dan $M$ adalah jumlah jenis koin. Kompleksitas ruangnya adalah $\mathcal{O}\left(N\right)$.


\subsection{Pseudocode}
\begin{verbatim}
    function coinChange(coins, amount):
        dp = array of size (amount + 1) initialized to infinity
        dp[0] = 0

        for i from 1 to amount:
            for each coin in coins:
                if i - coin >= 0:
                    dp[i] = min(dp[i], dp[i - coin] + 1)

        return dp[amount] if dp[amount] != infinity else -1
\end{verbatim}

\subsection{Contoh Input-Output}
\textbf{Input:}
\begin{verbatim}
    coins:
    1 3 4
    amount:
    6
\end{verbatim}

\textbf{Output:}
\begin{verbatim}
    minimum number of coins: 2
\end{verbatim}

\section{String Matching - Naive}
\label{sec:string-matching-implementation}

\subsection{Deskripsi Algoritma}
Naive String Matching adalah algoritma yang digunakan untuk mencari substring dalam string utama dengan cara membandingkan setiap karakter dari substring dengan karakter yang sesuai dalam string utama. Algoritma ini bekerja dengan iterasi melalui string utama dan memeriksa kecocokan karakter satu per satu. Kompleksitas waktu dalam kasus terburuk adalah $\mathcal{O}\left(N \cdot M\right)$, di mana $N$ adalah panjang string utama dan $M$ adalah panjang substring. Kompleksitas ruangnya adalah $\mathcal{O}(1)$.

\subsection{Pseudocode}
\begin{verbatim}
    function stringMatch(text, pattern):
        n = length of text
        m = length of pattern
        for i from 0 to n - m:
            j = 0
            while j < m and text[i + j] == pattern[j]:
                j += 1
            if j == m:
                return i  // found match at index i
        return -1  // no match found
\end{verbatim}

\subsection{Contoh Input-Output}
\textbf{Input:}
\begin{verbatim}
    text: "hello world"
    pattern: "world"
\end{verbatim}

\textbf{Output:}
\begin{verbatim}
    match found at index: 6
\end{verbatim}

\section{Path-Finding Algorithm - Dijkstra's Algorithm}
\label{sec:dijkstra-implementation}

\subsection{Deskripsi Algoritma}
Tuliskan penjelasan singkat tentang algoritma tersebut: cara kerja, kompleksitas waktu/ruang jika perlu, dan alasan penggunaannya secara umum.
Algoritma Dijkstra adalah algoritma yang digunakan untuk menemukan jalur terpendek dari satu vertex ke semua vertex lainnya dalam graf berbobot. Algoritma ini bekerja dengan cara menginisialisasi jarak dari simpul awal ke semua simpul lain sebagai tak hingga, kemudian secara iteratif memperbarui jarak terpendek yang ditemukan. Kompleksitas waktu Algoritma Dijkstra adalah $\mathcal{O}\left((V + E) \log V\right)$ dengan menggunakan priority queue, di mana $V$ adalah jumlah vertex dan $E$ adalah jumlah edge. Kompleksitas ruangnya adalah $\mathcal{O}\left(V + E\right)$.

\subsection{Pseudocode}
\begin{verbatim}
    function dijkstra(graph, start):
        dist = array of size V initialized to infinity
        dist[start] = 0
        priorityQueue = min-heap containing (0, start)

        while priorityQueue is not empty:
            (currentDist, currentVertex) = extract-min(priorityQueue)
            for each neighbor of currentVertex:
                newDist = currentDist + weight(currentVertex, neighbor)
                if newDist < dist[neighbor]:
                    dist[neighbor] = newDist
                    update priorityQueue with (newDist, neighbor)

        return dist
\end{verbatim}

\subsection{Contoh Input-Output}
\textbf{Input:}
\begin{verbatim}
    vertex count: 5
    edge count: 6
    edges: // (u, v, weight)
    1 2 2
    1 3 4
    2 3 1
    2 4 7
    3 5 3
    4 5 1
    start vertex: 1
\end{verbatim}

\textbf{Output:}
\begin{verbatim}
    shortest distances from source vertex 1:
    vertex 1: 0
    vertex 2: 2
    vertex 3: 3
    vertex 4: 7
    vertex 5: 6
\end{verbatim}

\section{Data Compression - Huffman Coding}
\label{sec:huffman-coding-implementation}

\subsection{Deskripsi Algoritma}
Tuliskan penjelasan singkat tentang algoritma tersebut: cara kerja, kompleksitas waktu/ruang jika perlu, dan alasan penggunaannya secara umum.
Huffman Coding adalah algoritma kompresi data yang digunakan untuk mengurangi ukuran data dengan cara menggantikan karakter yang sering muncul dengan kode biner yang lebih pendek. Algoritma ini membangun Huffman tree berdasarkan frekuensi kemunculan karakter dalam data. Kompleksitas waktu untuk membangun pohon Huffman adalah $\mathcal{O}\left(N \log N\right)$, di mana $N$ adalah jumlah karakter unik, dan kompleksitas ruangnya adalah $\mathcal{O}\left(N\right)$.

\subsection{Pseudocode}
\begin{verbatim}
    function huffmanCoding(frequencies):
        create a priority queue Q
        for each character and frequency in frequencies:
            insert (frequency, character) into Q

        while Q has more than one element:
            (freq1, char1) = extract-min(Q)
            (freq2, char2) = extract-min(Q)
            newNode = createNode(freq1 + freq2, char1 + char2)
            insert newNode into Q

        return root of the Huffman tree

    procedure generateCodes(node, code, codes):
        if node is a leaf:
            codes[node.character] = code
            return
        generateCodes(node.left, code + '0', codes)
        generateCodes(node.right, code + '1', codes)
\end{verbatim}

\subsection{Contoh Input-Output}
\textbf{Input:}
\begin{verbatim}
    text to compress: "aaaabbbccd"
\end{verbatim}

\textbf{Output:}
\begin{verbatim}
    Huffman Codes:
    'a': 0
    'b': 10
    'c': 111
    'd': 110
    Compressed text:
    "0000101010111111110"
\end{verbatim}

% \section{[Nama Algoritma]}

% \subsection{Deskripsi Algoritma}
% Tuliskan penjelasan singkat tentang algoritma tersebut: cara kerja, kompleksitas waktu/ruang jika perlu, dan alasan penggunaannya secara umum.

% \subsection{Pseudocode}
% \begin{verbatim}
% [Tulis pseudocode di sini dalam gaya umum, tidak spesifik bahasa]
% \end{verbatim}

% \subsection{Contoh Input-Output}
% \textbf{Input:}
% \begin{verbatim}
% [Tulis input seperti yang kamu uji di CLI]
% \end{verbatim}

% \textbf{Output:}
% \begin{verbatim}
% [Tulis hasil yang ditampilkan oleh program]
% \end{verbatim}

% sorting (bubble sort)
% perkalian matrix (naive matrix multiplication)
% integral numerik
% graph traversal (DFS, BFS)
% knapsack problem
% string matching
% path-finding algorithm
% huffman coding

\clearchapter
\include{src/01-body/bab5}
\clearchapter
\include{src/01-body/kesimpulan}
\clearchapter

%
% Daftar Pustaka
\CAPinToC % All entries in ToC will be CAPITALIZED from here on
\include{_internals/pustaka}
\clearchapter
\noCAPinToC % Revert to original \addcontentsline formatting


%
% Lampiran
%
\begin{appendix}
	\newcounter{pagetemp}
	\setcounter{pagetemp}{\thepage}
	\include{_internals/markLampiran}
	\clearchapter
	\setcounter{page}{\thepagetemp}
	\stepcounter{page}
	\include{src/99-backMatter/lampiran}
\end{appendix}

\end{document}
